\section{Kiến trúc công nghệ}
\subsection{Công cụ xử lý ngôn ngữ tự nhiên - Framework tích hợp LLMs}
Trong quá trình phát triển một hệ thống học tập trực tuyến thông minh dựa trên mô hình ngôn ngữ lớn (LLMs), việc lựa chọn công cụ xử lý ngôn ngữ tự nhiên phù hợp là một yếu tố quyết định đến hiệu quả và tính linh hoạt của hệ thống. Với mục tiêu tối ưu hóa trải nghiệm học tập cá nhân hóa, hệ thống cần tích hợp một framework mạnh mẽ để khai thác và xử lý dữ liệu từ nhiều nguồn. Hai công cụ nổi bật trong lĩnh vực này là LangChain và LlamaIndex, đều được thiết kế để hỗ trợ xây dựng các ứng dụng sử dụng LLMs. Báo cáo này sẽ phân tích và so sánh hai framework trên, nhằm xác định công cụ phù hợp nhất cho hệ thống học tập thông minh, dựa trên tính năng, khả năng tích hợp, và yếu tố chi phí.
\subsubsection{LangChain}
Langchain là một thư viện Python được thiết kế để hỗ trợ phát triển các ứng dụng dựa trên mô hình ngôn ngữ lớn (LLMs).
\par 
\textbf{\textit{Các tính năng chính:}}
\begin{itemize}
    \item Tích hợp dễ dàng với nhiều LLMs
    \begin{itemize}
        \item Dễ dàng tích hợp và chuyển đổi giữa các mô hình ngôn ngữ lớn khác nhau như \textbf{\textit{GPT-3, GPT-4, BERT}}, hoặc các mô hình tùy chỉnh.
        \begin{itemize}
            \item VD: Sử dụng \textbf{\textit{GPT-4}} cho việc tạo nội dung học tập phức tạp, trong khi sử dụng một mô hình nhẹ hơn như \textbf{\textit{BERT}} cho việc phân tích nhanh câu trả lời của học viên
        \end{itemize}
    \end{itemize}
    \item Chuỗi xử lý linh hoạt cho các tác vụ phức tạp
        \begin{itemize}
        \item Cho phép xây dựng các chuỗi xử lý (chains) phức tạp, kết hợp nhiều bước xử lý khác nhau.
        \begin{itemize}
            \item VD: Trong hệ thống gia sư AI, tạo một chuỗi xử lý để phân tích code của học viên, đánh giá, và đưa ra gợi ý cải thiện. 
        \end{itemize}
    \end{itemize}
    \item Hỗ trợ truy xuất thông tin và tìm kiếm ngữ nghĩa
        \begin{itemize}
        \item Cung cấp các công cụ để truy xuất thông tin dựa trên ngữ nghĩa, giúp tìm kiếm thông tin liên quan một cách hiệu quả
        \begin{itemize}
            \item VD: Khi học viên đặt câu hỏi, hệ thống có thể tìm kiếm thông tin liên quan từ cơ sở dữ liệu bài giảng bằng cách dùng \textbf{\textit{FAISS (Facebook AI Similarity Search)}} - Một thư viện hỗ trợ sử dụng vector search để tìm kiếm thông tin trong \textbf{\textit{Knowledge Base}}.
        \end{itemize}
    \end{itemize}
    \item Khả năng xử lý đa dạng các loại dữ liệu
        \begin{itemize}
        \item Có thể làm việc với nhiều loại dữ liệu khác nhau như văn bản, PDF, hình ảnh (thông qua tích hợp với các mô hình xử lý hình ảnh).
        \begin{itemize}
            \item VD: Xử lý các tài liệu PDF và chuyển đổi chúng thành nội dung học tập.
        \end{itemize}
    \end{itemize}
\end{itemize}
\subsubsection{LlamaIndex}
LlamaIndex là một framework chuyên biệt cho việc xây dựng các ứng dụng AI với khả năng truy xuất dữ liệu mạnh mẽ.
\par
\textbf{\textit{Các tính năng chính:}}
\begin{itemize}
    \item Indexing và truy xuất dữ liệu hiệu quả
        \begin{itemize}
        \item Chuyên về việc tạo index cho dữ liệu lớn và truy xuất thông tin một cách nhanh chóng và hiệu quả.
        \begin{itemize}
            \item VD: Indexing toàn bộ nội dung khóa học và truy xuất thông tin liên quan.
        \end{itemize}
        \item Qúa trình Indexing trong LlamaIndex như sau:
        \begin{itemize}
            \item Data Connectors: Đây là các công cụ nhập dữ liệu từ nhiều nguồn khác nhau như API, PDF, cơ sở dữ liệu, hay các ứng dụng bên ngoài (Gmail, Notion, Airtable). Chúng chịu trách nhiệm thu thập dữ liệu và đưa vào hệ thống dưới dạng tài liệu thống nhất.
            \item Documents/Nodes: Documents là các container chứa dữ liệu từ các nguồn khác nhau, chẳng hạn như từ PDF, API hoặc cơ sở dữ liệu. Nodes là những phần nhỏ của tài liệu, được bổ sung metadata và các mối quan hệ, giúp tăng độ chính xác trong quá trình truy xuất dữ liệu.
            \item Data Indexes: Sau khi dữ liệu được nhập vào, LlamaIndex sẽ giúp tổ chức dữ liệu thành một định dạng có thể truy xuất. Quá trình này bao gồm việc phân tích, tạo các biểu diễn embedding, và suy luận metadata, tạo thành kho kiến thức để sử dụng sau này.
        \end{itemize}
        \end{itemize}
    \item Tích hợp với nhiều nguồn dữ liệu
            \begin{itemize}
        \item LlamaIndex có thể làm việc với nhiều nguồn dữ liệu khác nhau như file local, API, cơ sở dữ liệu.
        \begin{itemize}
            \item VD: Tích hợp dữ liệu từ nhiều nguồn để tạo nội dung khóa học.
        \end{itemize}
        \end{itemize}
    \item Hỗ trợ cho các truy vấn phức tạp
            \begin{itemize}
        \item LlamaIndex cho phép thực hiện các truy vấn phức tạp, kết hợp nhiều điều kiện và lọc kết quả.
        \begin{itemize}
            \item VD: Tìm kiếm bài học phù hợp dựa trên nhiều tiêu chí.
        \end{itemize}
        \end{itemize}
\end{itemize}
\subsubsection{Tổng quan}
\begin{table}[H]
\centering
\begin{tabular}{|p{3.5cm}|p{6cm}|p{6cm}|}
\hline
\textbf{Tiêu chí} & \textbf{LangChain} & \textbf{LlamaIndex} \\ \hline
\textbf{Overall} & Phát triển ứng dụng phức tạp linh hoạt, tích hợp LLM & Tập trung vào tìm kiếm và truy xuất thông tin từ tập dữ liệu lớn một cách nhanh chóng và chính xác \\ \hline
\textbf{Prompts} & Hỗ trợ giao diện tiêu chuẩn cho việc tạo và quản lý prompts, giúp tùy chỉnh và sử dụng lại dễ dàng hơn trên các mô hình khác & Không hỗ trợ chi tiết \\ \hline
\textbf{Chains} & Cung cấp giao diện mạnh mẽ để xây dựng và quản lý chuỗi, cùng với nhiều thành phần có thể tái sử dụng & Không hỗ trợ chuỗi xử lý phức tạp \\ \hline
\textbf{Agents} & Sử dụng LLM để xác định và thực hiện các hành động dựa trên input & Không có cơ chế agent \\ \hline
\textbf{Data Indexing} & Phương pháp indexing qua các chuỗi phức tạp & Lập chỉ mục nhanh chóng các dữ liệu không cấu trúc \\ \hline
\textbf{Customization} & Tùy chỉnh cao cho các workflow và ứng dụng phức tạp & Tùy chỉnh hạn chế, tập trung vào lập chỉ mục và tìm kiếm \\ \hline
\textbf{Context Retention} & Lưu giữ thông tin từ các tương tác trước đó để cho phép các cuộc hội thoại có nhận thức về ngữ cảnh và mạch lạc & Lưu giữ ngữ cảnh cơ bản, phù hợp với nhiệm vụ tìm kiếm \\ \hline
\textbf{Use Cases} & Phù hợp cho các ứng dụng như chatbot, hỗ trợ khách hàng, tạo nội dung phức tạp & Phù hợp cho hệ thống tìm kiếm nội bộ, quản lý kiến thức \\ \hline
\textbf{Performance} & Xử lý tốt các cấu trúc dữ liệu phức tạp & Tối ưu cho tốc độ và độ chính xác trong truy xuất thông tin \\ \hline
\textbf{Lifecycle Management} & Cung cấp bộ đánh giá LangSmith để kiểm tra và gỡ lỗi ứng dụng LLM & Tích hợp công cụ gỡ lỗi và giám sát cho hiệu suất và độ tin cậy \\ \hline
\end{tabular}
\caption{So sánh giữa LangChain và LlamaIndex}
\end{table}

Với những ưu điểm trên, \textbf{LangChain} là một lựa chọn tối ưu cho hệ thống vì nó hỗ trợ mạnh mẽ về việc quản lý prompts – một yếu tố quan trọng để tạo ra các hướng dẫn cho các mô hình ngôn ngữ lớn (LLM), cho phép tuỳ chỉnh và tối ưu chức năng Adaptive Learning Path Management. Ngoài ra, tính năng memory của LangChain giúp lưu giữ và quản lý ngữ cảnh của các cuộc hội thoại trước đó, giúp các tương tác trở nên liền mạch và cá nhân hóa hơn.

\subsection{Framework xây dựng giao diện người dùng}

Framework xây dựng giao diện người dùng (UI Framework) giúp phát triển các ứng dụng web với giao diện người dùng linh hoạt và dễ dàng tương tác. Việc lựa chọn framework xây dựng giao diện người dùng (UI) là một quyết định quan trọng đối với các ứng dụng web hiện đại. React, Vue, và Angular là ba framework phổ biến nhất hiện nay, mỗi framework có điểm mạnh và hạn chế riêng. Báo cáo này sẽ so sánh ba framework này dựa trên các tiêu chí như tính dễ học, hiệu suất, khả năng mở rộng và sự linh hoạt trong phát triển.

\subsubsection{React}
React là một thư viện JavaScript được phát triển bởi Facebook, chuyên về việc xây dựng UI dựa trên các component tái sử dụng. React nổi bật với khả năng làm việc nhanh và hiệu quả nhờ vào Virtual DOM và hỗ trợ mạnh mẽ từ cộng đồng.

\par \textbf{\textit{Các tính năng chính:}} \begin{itemize} \item Virtual DOM giúp tối ưu hiệu suất render và cập nhật giao diện. \item Hệ sinh thái rộng lớn với các thư viện hỗ trợ như React Router, Redux, và React Query. \item Được tối ưu hóa cho các ứng dụng quy mô lớn với khả năng mở rộng mạnh mẽ. \item Yêu cầu kiến thức về JSX (JavaScript XML) và quản lý trạng thái với Redux hoặc Context API. \end{itemize}

\subsubsection{Vue}
Vue là một framework JavaScript nhẹ, dễ học, và được thiết kế để xây dựng các ứng dụng web động. Vue đặc biệt phù hợp với các nhà phát triển mới, với cú pháp đơn giản và dễ sử dụng, nhưng vẫn mạnh mẽ cho các ứng dụng phức tạp.

\par \textbf{\textit{Các tính năng chính:}} \begin{itemize} \item Cung cấp Vue CLI giúp khởi tạo dự án và xây dựng ứng dụng nhanh chóng. \item Hệ thống component mạnh mẽ và dễ hiểu. \item Vuex cho quản lý trạng thái đơn giản và dễ sử dụng. \item Tích hợp dễ dàng với các dự án hiện tại và khả năng mở rộng cho các ứng dụng lớn. \item Vue Router giúp quản lý routing trong các ứng dụng một cách dễ dàng. \end{itemize}

\subsubsection{Angular}
Angular là một framework full-stack được phát triển bởi Google, được biết đến với khả năng xây dựng các ứng dụng web phức tạp và quy mô lớn. Angular đi kèm với tất cả các công cụ cần thiết để xây dựng một ứng dụng đầy đủ chức năng.

\par \textbf{\textit{Các tính năng chính:}} \begin{itemize} \item Cung cấp một framework toàn diện, bao gồm công cụ để quản lý routing, form, HTTP requests, và các tính năng bảo mật. \item Tích hợp TypeScript giúp nâng cao tính an toàn của mã nguồn. \item Hệ thống dependency injection và mô hình MVC (Model-View-Controller) giúp tổ chức mã nguồn và quản lý các phần tử trong ứng dụng. \item Phức tạp và yêu cầu học hỏi nhiều hơn so với React và Vue, đặc biệt là khi sử dụng RxJS và các concepts như modules, services và directives. \end{itemize}

\subsubsection{Tổng quan}
\begin{table}[H]
    \centering
    \begin{tabular}{|p{2.5cm}|p{4.5cm}|p{4.5cm}|p{4.5cm}|}
    \hline
    \textbf{Tiêu chí} & \textbf{React} & \textbf{Vue} & \textbf{Angular} \\
    \hline
    \textbf{Dễ học} & Dễ học với JSX và các concept như state và lifecycle nhưng yêu cầu làm quen với các thư viện như Redux & Dễ học với cú pháp đơn giản, dễ tiếp cận và tài liệu phong phú & Phức tạp hơn, yêu cầu học các khái niệm như Dependency Injection, RxJS và TypeScript \\
    \hline
    \textbf{Hiệu suất} & Rất tốt nhờ Virtual DOM và khả năng render nhanh chóng & Cũng sử dụng Virtual DOM nhưng nhẹ và nhanh hơn trong các ứng dụng nhỏ và trung bình & Hiệu suất ổn, nhưng thường chậm hơn trong các ứng dụng lớn do tính phức tạp của framework \\
    \hline
    \textbf{Quản lý trạng thái} & Quản lý trạng thái mạnh mẽ nhưng phức tạp với Redux hoặc Context API & Quản lý trạng thái đơn giản với Vuex, dễ sử dụng cho cả ứng dụng nhỏ và lớn & Quản lý trạng thái mạnh mẽ với NgRx, nhưng phức tạp hơn Vuex và Redux \\
    \hline
    \textbf{Khả năng mở rộng} & Phù hợp cho các ứng dụng quy mô lớn và có thể mở rộng dễ dàng nhờ vào hệ sinh thái thư viện mạnh mẽ & Rất linh hoạt và có thể mở rộng cho cả các ứng dụng nhỏ và lớn mà không gặp phải sự phức tạp như Angular & Tốt cho các ứng dụng quy mô lớn với tính cấu trúc chặt chẽ và tổ chức mã nguồn rõ ràng \\
    \hline
    \textbf{Cộng đồng} & Cộng đồng lớn với nhiều tài nguyên và thư viện hỗ trợ & Cộng đồng đang phát triển, với tài liệu dễ hiểu và hỗ trợ từ cộng đồng ngày càng mạnh mẽ & Cộng đồng lớn, nhưng tài liệu có thể phức tạp và khó tiếp cận hơn đối với người mới bắt đầu \\
    \hline
    \textbf{Tính linh hoạt} & Linh hoạt cao, nhưng yêu cầu sử dụng các thư viện bổ sung cho các tính năng như routing và quản lý trạng thái & Linh hoạt và dễ dàng tích hợp vào các dự án hiện tại, cung cấp tính năng đầy đủ mà không cần thư viện bên ngoài & Tính linh hoạt thấp hơn so với React và Vue, yêu cầu tuân theo một cấu trúc chặt chẽ và sử dụng các công cụ riêng biệt \\
    \hline
    \end{tabular}
    \caption{So sánh giữa React, Vue và Angular}
\end{table}
Với cú pháp dễ sử dụng, khả năng mở rộng tốt và sự dễ dàng trong việc tích hợp vào dự án hiện tại, nhóm quyết định chọn Vue.js làm framework xây dựng giao diện người dùng cho hệ thống học tập thông minh.
\subsection{Framework phát triển API}

Framework phát triển API là công cụ quan trọng trong việc xây dựng các hệ thống web hiện đại. Việc lựa chọn framework phát triển API là rất quan trọng trong các ứng dụng web và dịch vụ backend. FastAPI và Django đều có những tính năng mạnh mẽ và phù hợp cho việc phát triển API, nhưng mỗi công cụ lại có những ưu điểm riêng. Báo cáo này sẽ so sánh FastAPI và Django dựa trên hiệu suất, tính dễ sử dụng, và khả năng mở rộng.

\subsubsection{FastAPI}
FastAPI là một framework web nhanh chóng được phát triển bằng Python, được tối ưu hóa cho việc xây dựng các API với tính bảo mật cao và dễ dàng mở rộng.

\par \textbf{\textit{Các tính năng chính:}} \begin{itemize} \item Tốc độ cao nhờ vào Starlette và Pydantic \item Tự động tạo tài liệu API với OpenAPI \item Hỗ trợ đầy đủ các chuẩn như HTTP, WebSockets, GraphQL, và OAuth2 \end{itemize}

\subsubsection{Django}
Django là một framework web Python nổi tiếng, cung cấp đầy đủ các tính năng để phát triển ứng dụng web bao gồm API, hệ quản trị cơ sở dữ liệu và bảo mật.

\par \textbf{\textit{Các tính năng chính:}} \begin{itemize} \item Cung cấp các công cụ tích hợp như Django REST Framework (DRF) để xây dựng API \item Quản lý cơ sở dữ liệu mạnh mẽ với ORM tích hợp sẵn \item Hệ sinh thái lớn với nhiều thư viện hỗ trợ \end{itemize}

\subsubsection{Tổng quan}
\begin{table}[H]
    \centering
    \begin{tabular}{|p{3cm}|p{6.5cm}|p{6.5cm}|}
    \hline
    \textbf{Tiêu chí} & \textbf{FastAPI} & \textbf{Django} \\
    \hline
    \textbf{Hiệu suất} & Rất nhanh nhờ vào Starlette và Pydantic & Hiệu suất tốt nhưng không thể so sánh với FastAPI trong các ứng dụng quy mô lớn \\
    \hline
    \textbf{Dễ sử dụng} & Dễ học, đặc biệt cho người đã quen Python & Được hỗ trợ tốt với tài liệu phong phú và cộng đồng lớn \\
    \hline
    \textbf{Tính bảo mật} & Tích hợp đầy đủ các công cụ bảo mật hiện đại như OAuth2, JWT & Cung cấp các công cụ bảo mật mạnh mẽ nhưng đòi hỏi cấu hình thêm \\
    \hline
    \textbf{Khả năng mở rộng} & Rất dễ mở rộng và thích hợp cho các ứng dụng hiện đại & Hỗ trợ mở rộng tốt với kiến trúc có sẵn \\
    \hline
    \end{tabular}
    \caption{So sánh giữa FastAPI và Django}
\end{table}
Với hiệu suất cao, tính năng tự động tạo tài liệu API và khả năng dễ dàng tích hợp, nhóm quyết định chọn FastAPI làm framework phát triển API cho hệ thống học tập thông minh.
\section{Tổng kết}
\par Chương 2 đã cung cấp một cái nhìn tổng quan về các khái niệm và lý thuyết nền tảng trong lĩnh vực giáo dục và giáo dục thông minh. Chúng ta đã tìm hiểu về khái niệm giáo dục, vai trò quan trọng của nó trong việc phát triển toàn diện con người và sự thích nghi với môi trường xã hội. Đồng thời, khái niệm giáo dục thông minh đã được làm rõ, nhấn mạnh vào sự kết hợp giữa công nghệ và các phương pháp học tập cá nhân hóa, nhằm nâng cao hiệu quả giảng dạy và học tập.

\par Các khái niệm về mô hình ngôn ngữ, từ mô hình ngôn ngữ xác suất đến các mô hình ngôn ngữ lớn (LLMs), đã được trình bày chi tiết, làm nền tảng để hiểu rõ hơn về công nghệ ứng dụng trong giáo dục thông minh. Đặc biệt, những thuật ngữ quan trọng như đồ thị tri thức (Knowledge Graph), Knowledge Tracing, và các công nghệ hỗ trợ trong giáo dục như LangChain, LlamaIndex, Neo4J, và VueJS đã được giới thiệu, nhằm làm rõ các công cụ và nền tảng công nghệ cần thiết để phát triển hệ thống học tập thông minh hiện đại.

\par Kết thúc chương này, ta có thể thấy rằng việc áp dụng công nghệ vào giáo dục không chỉ là một xu hướng mà còn là một yêu cầu thiết yếu để nâng cao chất lượng và hiệu quả học tập. Trong chương tiếp theo, chúng ta sẽ đi sâu vào việc thiết kế và triển khai hệ thống giáo dục thông minh, đồng thời đánh giá các yếu tố ảnh hưởng đến sự thành công của hệ thống này.
\chapter{Đánh giá hệ thống}

\section{Đánh giá tính năng và hiệu quả của hệ thống}

Hệ thống hỗ trợ học tập bằng cách sử dụng LLMs (Language Models) để tự động tạo ra các bài học đề xuất cho sinh viên dựa trên các bài học có sẵn từ giảng viên. Các tính năng đã triển khai, như trang dashboard, course list, course detail, và các trang liên quan đến việc đề xuất lessons đã bước đầu cho phép sinh viên tương tác hiệu quả với hệ thống học tập.

Trong quá trình phát triển, nhóm đã hoàn thành việc xây dựng giao diện và chức năng cơ bản, bao gồm:

\begin{itemize}
    \item \textbf{Dashboard}: Cung cấp một cái nhìn tổng quan về các hoạt động học tập gần đây, tình trạng tiến độ học tập của sinh viên và các khóa học tham gia.
    \item \textbf{Course List và Course Detail}: Cho phép sinh viên theo dõi các khóa học, xem chi tiết các bài học, và tiếp cận với các tài liệu học tập.
    \item \textbf{Đề xuất Lessons}: Dựa trên mục tiêu học tập mà sinh viên đặt ra, hệ thống sẽ tự động tìm kiếm các bài học phù hợp từ các khóa học đã đăng tải, giúp sinh viên tiếp cận đúng bài học cần thiết.
\end{itemize}

Các mô-đun học tập (quiz, code exercises, reading material) đã được phát triển cho phép sinh viên tham gia vào việc học theo dạng quiz, code exercises, và tài liệu đọc. Hệ thống cũng đã ghi lại quá trình học tập của sinh viên, tạo điều kiện để tiến hành báo cáo và đánh giá kết quả học tập.
\newpage
\section{Đánh giá các hệ thống liên quan}
\begin{table}[ht]
    \resizebox{\textwidth}{!}{
    \begin{tabular}{|p{3.5cm}|p{3.5cm}|p{3.5cm}|p{3.5cm}|p{3.5cm}|p{3.5cm}|p{3.5cm}|}
    \hline
    \textbf{Tiêu chí} & \textbf{Hệ thống Adaptive LearningPath  \cite{Raj2022}} & \textbf{Hệ thống FOKE \cite{hu2024fokepersonalizedexplainableeducation}} & \textbf{CS50.ai\cite{10.1145/3626252.3630938}} & \textbf{APAS\cite{Frankford_2024}} & \textbf{LeetCode} \\ \hline
    \textbf{Cá nhân hóa} & Đề xuất lộ trình dựa trên năng lực học sinh & Dựa trên hồ sơ người học và kỹ thuật prompt engineering & Gợi ý mã nguồn, trợ lý học tập AI & Phản hồi tức thời, gợi ý cải thiện mã & Tạo bài tập nhưng ít cá nhân hóa \\ \hline
    \textbf{Tính tương thích với người dùng} & Thông qua đồ thị tri thức và tài nguyên phù hợp & Tùy chỉnh theo phong cách học tập và hành vi & Trợ lý học tập dựa trên GPT-4, trả lời câu hỏi học thuật & Phản hồi chi tiết về lỗi và cải tiến mã & Tạo bài tập dựa trên ngữ cảnh nhưng ít tùy chỉnh \\ \hline
    \textbf{Khả năng tương tác} & Chỉ có gợi ý tài liệu học tập & Tương tác qua gợi ý cấu trúc và phản hồi liên tục & Chatbot tương tác trực tiếp với sinh viên & AI-Tutor cung cấp phản hồi và gợi ý cải thiện mã & Tương tác chủ yếu thông qua bài tập và câu hỏi \\ \hline
    \textbf{Độ chính xác của đề xuất} & Tùy chỉnh theo dữ liệu thực tế & Sử dụng graph embedding và gợi ý cấu trúc & AI đưa ra phản hồi chính xác về mã nguồn & Phản hồi tức thời về mã nguồn và gợi ý cải thiện & Tạo bài tập nhưng không có giải thích chi tiết \\ \hline
    \textbf{Sự minh bạch và giải thích} & Đưa ra lý do thay đổi lộ trình học & Cung cấp giải thích rõ ràng về lộ trình học & Giải thích chi tiết về mã nguồn và phong cách lập trình & Cung cấp lý do cải thiện mã, không tiết lộ đáp án & Ít giải thích về cách tạo bài tập hoặc câu hỏi \\ \hline
    \textbf{Tính mở rộng} & Có thể mở rộng với nhiều tài nguyên học tập & Có thể mở rộng với nhiều đối tượng người học và lĩnh vực & Có thể áp dụng cho các khóa học online lớn & Có thể mở rộng cho nhiều khóa học và bài tập & Hệ thống có sẵn các bài tập cho nhiều cấp độ \\ \hline
    \textbf{Ứng dụng vào thực tế} & Có thể áp dụng cho lớp học truyền thống và trực tuyến & Ứng dụng vào nhiều môi trường học tập khác nhau & Ứng dụng trong các khóa học lớn như CS50 & Hệ thống đánh giá tự động có thể áp dụng cho lớp học và MOOCs & Ứng dụng cho các bài tập lập trình cá nhân và nhóm \\ \hline
    \textbf{Tiết kiệm thời gian cho giảng viên} & Giảm tải công việc của giảng viên & Giảm tải cho giảng viên trong việc thiết kế lộ trình học & Giảm tải cho giảng viên trong việc giải thích mã & Giảm tải cho giảng viên trong việc kiểm tra và đánh giá bài tập & Giảm tải cho giảng viên trong việc tạo bài tập \\ \hline
    \textbf{Hiệu quả học tập} & Cải thiện khả năng học của học sinh qua lộ trình cá nhân hóa & Cải thiện hiệu quả học tập nhờ cá nhân hóa sâu & Tăng cường khả năng tự học của sinh viên & Giúp sinh viên phát triển tư duy thuật toán & Hiệu quả học tập phụ thuộc vào việc tự học của người dùng \\ \hline
    \end{tabular}
    }
    \caption{Đánh giá các hệ thống liên quan}
    \end{table}
\begin{itemize}
    \item Các tiêu chí đánh giá hệ thống
    \begin{enumerate}
        \item Cá nhân hóa: Đánh giá mức độ cá nhân hóa của lộ trình học tập, phản hồi và tài nguyên học tập.
        \item Tính tương thích với người dùng: Đánh giá khả năng tương thích của hệ thống với các đặc điểm người dùng như phong cách học tập, mục tiêu học tập, và năng lực hiện tại.
        \item Khả năng tương tác: Đánh giá mức độ và tính hiệu quả của sự tương tác giữa người học và hệ thống (ví dụ: chatbot, phản hồi tức thời).
        \item Độ chính xác của đề xuất: Đánh giá khả năng của hệ thống trong việc tạo ra các đề xuất chính xác, phù hợp với người học.
        \item Sự minh bạch và giải thích: Đánh giá mức độ giải thích rõ ràng của hệ thống về các đề xuất và lý do chọn các tài liệu học tập.
        \item Tính mở rộng: Đánh giá khả năng mở rộng của hệ thống khi có nhiều người học và tài nguyên giáo dục hơn.
        \item Ứng dụng vào thực tế: Đánh giá khả năng áp dụng hệ thống vào môi trường học tập thực tế, như lớp học truyền thống, lớp học trực tuyến, hoặc hệ thống MOOCs.
        \item Tiết kiệm thời gian cho giảng viên: Đánh giá mức độ mà hệ thống giúp giảng viên giảm tải công việc như đánh giá bài tập, phản hồi học sinh, v.v.
        \item Hiệu quả học tập: Đánh giá mức độ cải thiện kết quả học tập của sinh viên khi sử dụng hệ thống.
    \end{enumerate}
\end{itemize}
\section{Các vấn đề và khó khăn trong quá trình phát triển}

Mặc dù hệ thống đã đạt được nhiều tiến triển, nhưng quá trình phát triển không thiếu thử thách. Một số vấn đề chính bao gồm:

\begin{itemize}
    \item \textbf{Khó khăn trong việc đồng bộ công nghệ}: Một số thành viên trong nhóm chưa thành thạo Docker, điều này gây khó khăn trong việc đồng bộ mã nguồn và triển khai hệ thống trên môi trường phát triển chung. Việc thiếu sự đồng thuận trong cách sử dụng các công cụ công nghệ cũng làm chậm tiến độ và dễ dẫn đến sự không nhất quán trong cấu hình hệ thống.
    \item \textbf{Vấn đề về thời gian của các thành viên}: Một số thành viên trong nhóm có lịch trình không đồng đều, điều này dẫn đến việc không thể hoàn thành các phần việc đúng hạn. Mặc dù đội ngũ đã cố gắng chia sẻ công việc hợp lý, nhưng sự thiếu hụt nhân lực vào những giai đoạn quan trọng khiến công việc bị trễ.
    \item \textbf{Chưa hoàn thiện phần chức năng Authentication}: Hiện tại, chức năng authentication của hệ thống chưa được hiện thực, khiến cho việc quản lý người dùng và bảo mật hệ thống chưa được đảm bảo. Điều này sẽ là một yếu tố quan trọng trong giai đoạn tiếp theo của hệ thống.
    \item \textbf{Chưa có kinh nghiệm kiểm thử chuyên nghiệp}: Mặc dù chúng tôi đã thực hiện một số kiểm thử API cơ bản, nhưng đội ngũ thiếu kinh nghiệm kiểm thử chuyên nghiệp và đầy đủ. Các khía cạnh khác như kiểm thử hiệu suất, giao diện người dùng (UI), và các thành phần của hệ thống vẫn chưa được kiểm tra một cách toàn diện, điều này có thể ảnh hưởng đến độ ổn định và hiệu quả của hệ thống trong các tình huống thực tế.
    \item \textbf{Tồn tại những testcase APIs failed trong quá trình phát triển}: Một số testcase APIs vẫn chưa hoàn thiện, điều này ảnh hưởng đến việc kiểm thử và đảm bảo chất lượng của hệ thống.
    \item \textbf{Chưa hoàn thiện tính năng tích hợp AI vào backend}: Mặc dù đã có sự nỗ lực trong việc tích hợp LLMs vào hệ thống, nhưng việc xây dựng mô-đun học tập cụ thể và sinh ra các đề xuất bài học vẫn chưa được hoàn thiện. Điều này ảnh hưởng đến khả năng cung cấp các bài học chất lượng và phù hợp với nhu cầu học tập của sinh viên.
\end{itemize}

Ngoài ra, việc tích hợp LLMs để sinh ra các đề xuất bài học và xây dựng mô-đun học tập cụ thể cũng gặp một số thách thức về hiệu suất và độ chính xác của các đề xuất. Cần cải thiện và tối ưu thuật toán để nâng cao chất lượng các bài học sinh ra.

\section{Kết chương}

Tổng thể, hệ thống học tập đã đạt được những kết quả bước đầu đáng ghi nhận, với nhiều tính năng hữu ích cho sinh viên. Tuy nhiên, còn nhiều vấn đề cần giải quyết, đặc biệt là liên quan đến việc đồng bộ công nghệ, hoàn thiện chức năng authentication, và cải thiện các đề xuất học tập thông qua LLMs. Những vấn đề này sẽ được khắc phục trong các giai đoạn phát triển tiếp theo.


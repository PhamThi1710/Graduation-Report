\chapter{Đánh giá hệ thống}

\section{Đánh giá tính năng và hiệu quả của hệ thống}

\subsection{Tính năng chính dựa trên LLM}

Dưới đây là danh sách các tính năng chính đã được hoàn thành trong hệ thống học tập thông minh, đặc biệt nhấn mạnh vào các tính năng liên quan đến việc ứng dụng mô hình ngôn ngữ lớn (LLM) để tối ưu hóa trải nghiệm học tập:

\begin{table}[H]
\centering
\caption{Danh sách các tính năng đã hoàn thành trong hệ thống}
\begin{tabular}{|p{5cm}|p{1.5cm}|p{9.5cm}|}
\hline
\textbf{Tính năng} & \textbf{Hoàn thành} & \textbf{Mô tả chi tiết} \\
\hline
\multicolumn{3}{|c|}{\textbf{Đề xuất và cá nhân hóa lộ trình học tập}} \\
\hline
Phân tích mục tiêu học tập & \checkmark & Sử dụng LLM phân tích mục tiêu học tập của sinh viên theo nguyên tắc SMART (Specific, Measurable, Achievable, Relevant, Time-bound). \\
\hline
Sinh tự động lộ trình học tập & \checkmark & Hệ thống sẽ sinh ra lộ trình học tập cá nhân hóa dựa trên mục tiêu học tập, nội dung khóa học và thời gian có sẵn. \\
\hline
Gợi ý mục tiêu học tập & \checkmark & Cung cấp các gợi ý mục tiêu học tập cho sinh viên dựa trên khóa học, với ba cấp độ khác nhau: Struggling (khó khăn), Average (trung bình) và Advanced (nâng cao). \\
\hline
Điều chỉnh động lộ trình học tập & \checkmark & Cập nhật lộ trình học tập dựa trên tiến độ và kết quả học tập của sinh viên. \\
\hline
\multicolumn{3}{|c|}{\textbf{Tạo và đánh giá nội dung học tập}} \\
\hline
Tạo module học tập động & \checkmark & Tự động tạo các module học tập có cấu trúc rõ ràng với mục tiêu, nội dung đọc và bài tập thực hành. \\
\hline
Tạo bổ sung nội dung học tập & \checkmark & Phân tích các vấn đề học tập của sinh viên và tạo bổ sung nội dung học tập phù hợp, tập trung vào các khía cạnh cần cải thiện. \\
\hline
Sinh tự động bài kiểm tra & \checkmark & Tạo các bài kiểm tra với nhiều mức độ, số lượng, phân bố hợp lý giữa các loại câu hỏi (dễ, trung bình, khó). \\
\hline
Đánh giá tiến độ học tập & \checkmark & Đánh giá tiến độ học tập của sinh viên theo phương pháp Rubric-Based Assessment và STAR, với các tiêu chí cụ thể về kiến thức lý thuyết, kỹ năng thực hành và mức độ nỗ lực. \\
\hline
\multicolumn{3}{|c|}{\textbf{Hỗ trợ và tương tác thông minh}} \\
\hline
Gia sư AI & \checkmark & Cung cấp hỗ trợ trực tiếp cho sinh viên khi gặp khó khăn trong học tập, giải thích mã nguồn, gợi ý cải thiện, và hướng dẫn sửa lỗi. \\
\hline
Tạo bài tập lập trình & \checkmark & Tự động tạo bài tập lập trình với độ khó phù hợp, kèm theo test cases để đánh giá chính xác kết quả. \\
\hline
Phân tích kết quả học tập & \checkmark & Phân tích dữ liệu học tập của sinh viên để đưa ra các gợi ý cải thiện, với thuật toán phát hiện vấn đề học tập và khó khăn phổ biến. \\
\hline
\end{tabular}

\end{table}

\subsection{Các yêu cầu chức năng khác}
\begin{table}[H]
    \centering
    \caption{Danh sách các tính năng chức năng đã hoàn thành (Phần 1)}
    \begin{tabular}{|p{5cm}|p{1.5cm}|p{9.5cm}|}
    \hline
    \textbf{Tính năng} & \textbf{Hoàn thành} & \textbf{Mô tả chi tiết} \\
    \hline
    \multicolumn{3}{|c|}{\textbf{Quản lý tài khoản (User Management)}} \\
    \hline
    Đăng ký và xác thực tài khoản & \checkmark & Hỗ trợ đăng ký và xác thực tài khoản cho ba vai trò: Sinh viên, Giảng viên và Quản trị viên, với tích hợp đăng nhập qua email trường đại học và Google. \\
    \hline
    Làm mới token xác thực & \checkmark & Tự động làm mới token xác thực khi phiên đăng nhập hết hạn. \\
    \hline
    Quên và đặt lại mật khẩu & \checkmark & Hỗ trợ đầy đủ quy trình quên mật khẩu, gửi email xác nhận và đặt lại mật khẩu. \\
    \hline
    Cập nhật thông tin cá nhân & \checkmark & Cho phép người dùng cập nhật thông tin cá nhân như tên, ngày sinh, avatar và các thông tin khác. \\
    \hline
    \multicolumn{3}{|c|}{\textbf{Quản lý khóa học (Course Management)}} \\
    \hline
    Xem danh sách khóa học & \checkmark & Hiển thị danh sách khóa học đã đăng ký (sinh viên) hoặc phụ trách (giảng viên) có thể phân trang và tìm kiếm. \\
    \hline
    Xem chi tiết khóa học & \checkmark & Hiển thị thông tin chi tiết về khóa học, bao gồm số lượng sinh viên, bài học, bài tập, tài liệu và tiến độ học tập. \\
    \hline
    Cập nhật mục tiêu học tập & \checkmark & Cho phép giảng viên cập nhật mục tiêu học tập và thông tin khóa học, bao gồm ảnh đại diện khóa học. \\
    \hline
    Tạo và quản lý khóa học & \checkmark & Quản trị viên có thể tạo khóa học mới (đơn lẻ hoặc hàng loạt), cập nhật thông tin khóa học và xóa khóa học khi cần. \\
    \hline
    \multicolumn{3}{|c|}{\textbf{Quản lý bài học (Lesson Management)}} \\
    \hline
    Xem danh sách bài học & \checkmark & Sinh viên xem danh sách bài học trong khóa học, bao gồm thông tin về mục tiêu và tài liệu của mỗi bài học. \\
    \hline
    Tạo và quản lý bài học & \checkmark & Giảng viên có thể tạo bài học mới, cập nhật thông tin bài học và xóa bài học cùng tài liệu liên quan. \\
    \hline
    Quản lý tài liệu bài học & \checkmark & Đăng tải document vào bài học với nhiều định dạng khác nhau như PDF, PowerPoint, Word, v.v. \\
    \hline
    % \multicolumn{3}{|c|}{\textbf{Quản lý bài tập (Exercise Management)}} \\
    % \hline
    % Xem và làm bài tập & \checkmark & Sinh viên có thể xem danh sách, chi tiết bài tập và làm bài tập trực tiếp trên hệ thống. \\
    % \hline
    % Tạo và quản lý bài tập & \checkmark & Giảng viên có thể tạo, cập nhật và xóa bài tập dạng quiz hoặc lập trình với test cases tự động. \\
    % \hline
    % Nộp và đánh giá bài tập & \checkmark & Hỗ trợ nộp bài tập, chấm tự động và đánh giá kết quả với phản hồi chi tiết. \\
    % \hline
    \end{tabular}

    \end{table}
    \begin{table}[H]
        \centering
        \caption{Danh sách các tính năng chức năng đã hoàn thành (Phần 2)}
        \begin{tabular}{|p{5cm}|p{1.5cm}|p{9.5cm}|}
        \hline
        \textbf{Tính năng} & \textbf{Hoàn thành} & \textbf{Mô tả chi tiết} \\
        \hline
        \multicolumn{3}{|c|}{\textbf{Theo dõi tiến độ học tập (Progress Tracking)}} \\
        \hline
        Cập nhật thời gian học & \checkmark & Tự động ghi nhận và cập nhật thời gian học tập của sinh viên cho từng bài học đề xuất. \\
        \hline
        Xem phân tích tiến độ & \checkmark & Sinh viên có thể xem phân tích tiến độ học tập chi tiết trong khóa học hoặc bài học cụ thể. \\
        \hline
        Xem điểm số & \checkmark & Giảng viên có thể xem điểm số của sinh viên trong khóa học, mục tiêu và lộ trình học tập của sinh viên. \\
        \hline
        \multicolumn{3}{|c|}{\textbf{Quản lý phản hồi (Feedback Management)}} \\
        \hline
        Gửi phản hồi & \checkmark & Sinh viên và giảng viên có thể gửi phản hồi về hệ thống với đánh giá và mô tả. \\
        \hline
        Xem và quản lý phản hồi & \checkmark & Quản trị viên và giảng viên có thể xem danh sách phản hồi, lọc theo thời gian và trạng thái. \\
        \hline
        Cập nhật trạng thái phản hồi & \checkmark & Quản trị viên có thể cập nhật trạng thái phản hồi (đang chờ, đang xử lý, đã giải quyết). \\
        \hline
        \multicolumn{3}{|c|}{\textbf{Quản lý người dùng (User Administration)}} \\
        \hline
        Tạo người dùng mới & \checkmark & Quản trị viên có thể tạo tài khoản cho sinh viên, giảng viên và quản trị viên khác. \\
        \hline
        Quản lý danh sách người dùng & \checkmark & Hỗ trợ xem danh sách, tìm kiếm, lọc và quản lý thông tin chi tiết người dùng. \\
        \hline
        Quản lý trạng thái người dùng & \checkmark & Cho phép bật/tắt tài khoản người dùng để kiểm soát quyền truy cập vào hệ thống. \\
        \hline
        \multicolumn{3}{|c|}{\textbf{Dashboard và báo cáo (Dashboard \& Reporting)}} \\
        \hline
        Dashboard sinh viên & \checkmark & Hiển thị tổng quan về tiến độ học tập, khóa học gần đây và các hoạt động học tập. \\
        \hline
        Dashboard giảng viên & \checkmark & Cung cấp tổng quan về hoạt động giảng dạy, số lượng khóa học, bài học và sinh viên. \\
        \hline
        Dashboard quản trị viên & \checkmark & Hiển thị thông tin tổng quan về hệ thống, người dùng và hoạt động trên toàn hệ thống. \\
        \hline
        \end{tabular}

        \end{table}
        
\subsection{Các yêu cầu phi chức năng}

\begin{table}[H]
    \centering
    \caption{Danh sách các yêu cầu phi chức năng đã đáp ứng (Phần 1)}
    \begin{tabular}{|p{3cm}|p{1.5cm}|p{4.5cm}|p{7cm}|}
    \hline
    \textbf{Loại yêu cầu} & \textbf{Hoàn thành} & \textbf{Tiêu chí} & \textbf{Chi tiết triển khai} \\
    \hline
    \multirow{3}{*}{Hiệu năng} & X & Thời gian phản hồi nhanh & Thời gian phản hồi cho các thao tác thông thường dưới 2 giây. \\
    \cline{2-4}
    & Chưa đánh giá & Hỗ trợ nhiều người dùng & Hệ thống hỗ trợ tối thiểu 1000 người dùng đồng thời mà không giảm hiệu năng. \\
    \cline{2-4}
    & X \footnotemark& Thời gian phản hồi AI & Tối ưu hóa API calls tới các model LLM, giữ thời gian phản hồi AI dưới 8 giây cho các tác vụ phức tạp. \\
    \hline
    \multirow{2}{*}{Khả năng mở rộng} & \checkmark & Kiến trúc microservices & Thiết kế hệ thống theo kiến trúc module, dễ dàng mở rộng và thêm tính năng mà không ảnh hưởng đến thành phần khác. \\
    \cline{2-4}
    & \checkmark & Triển khai trên cloud & Sử dụng các dịch vụ cloud có khả năng auto-scaling. \\
    \hline
    \multirow{3}{*}{Bảo mật} & \checkmark & Mã hóa dữ liệu & Mã hóa dữ liệu cá nhân và thông tin học tập bằng chuẩn AES-256 trong quá trình lưu trữ và truyền tải. \\
    \cline{2-4}
    & \checkmark & Xác thực đa yếu tố & Triển khai xác thực đa yếu tố cho các tài khoản quản trị và giảng viên, tăng cường bảo mật. \\
    \cline{2-4}
    & \checkmark & Phân quyền chi tiết & Hệ thống phân quyền chi tiết theo vai trò Student, Professor, Admin. \\
    \hline
    \end{tabular}

    \end{table}
    \footnotetext{Chi tiết về thời gian phản hồi trung bình của các tính năng sử dụng LLM được trình bày trong Bảng \ref{table:llm_response_time}.}
    \begin{table}[H]
        \centering
        \caption{Danh sách các yêu cầu phi chức năng đã đáp ứng (Phần 2)}
        \begin{tabular}{|p{3cm}|p{1.5cm}|p{4.5cm}|p{7cm}|}
        \hline
        \textbf{Loại yêu cầu} & \textbf{Hoàn thành} & \textbf{Tiêu chí} & \textbf{Chi tiết triển khai} \\
        \hline
        \multirow{3}{*}{Tính khả dụng} & \checkmark & Giao diện thân thiện & Thiết kế giao diện trực quan, dễ sử dụng với bố cục rõ ràng và màu sắc hợp lý. \\
        \cline{2-4}
        & \checkmark & Tính nhất quán & Đảm bảo tính nhất quán trong thiết kế và trải nghiệm người dùng trên toàn bộ hệ thống. \\
        % \cline{2-4}
        % & \checkmark & Thiết kế đáp ứng & Giao diện đáp ứng (responsive) trên nhiều thiết bị và kích thước màn hình khác nhau. \\
        \hline
        \multirow{2}{*}{Tính tương thích} & \checkmark & Tương thích trình duyệt & Hỗ trợ các trình duyệt phổ biến (Chrome, Firefox, Safari, Edge) với trải nghiệm nhất quán. \\
        \cline{2-4}
        & \checkmark & Tương thích hệ điều hành & Hoạt động ổn định trên các hệ điều hành phổ biến (Windows, macOS, Linux) và thiết bị di động. \\
        \hline
        \multirow{2}{*}{Khả năng bảo trì} & \checkmark & Mã nguồn có cấu trúc & Mã nguồn được tổ chức theo cấu trúc rõ ràng, dễ đọc và tuân thủ các tiêu chuẩn coding. \\
        \cline{2-4}
        & \checkmark & Tài liệu đầy đủ & Hệ thống có tài liệu kỹ thuật đầy đủ, bao gồm API documentation và hướng dẫn triển khai. \\
        \hline
        \multirow{2}{*}{Độ tin cậy} & \checkmark & Tính ổn định cao & Hệ thống duy trì uptime ổn định đảm bảo trải nghiệm người dùng liên tục. \\
        \cline{2-4}
        & \checkmark & Sao lưu tự động & Dữ liệu được sao lưu tự động theo lịch trình, đảm bảo khả năng phục hồi nhanh chóng. \\
        \hline
        \end{tabular}

        \end{table}
% \section{Các vấn đề và khó khăn trong quá trình phát triển}

% \section{Kết chương}


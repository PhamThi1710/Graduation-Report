\chapter{Đánh giá hệ thống}

\section{Đánh giá tính năng và hiệu quả của hệ thống}

Hệ thống hỗ trợ học tập bằng cách sử dụng LLMs (Language Models) để tự động tạo ra các bài học đề xuất cho sinh viên dựa trên các bài học có sẵn từ giảng viên. Các tính năng đã triển khai, như trang dashboard, course list, course detail, và các trang liên quan đến việc đề xuất lessons đã bước đầu cho phép sinh viên tương tác hiệu quả với hệ thống học tập.

Trong quá trình phát triển, nhóm đã hoàn thành việc xây dựng giao diện và chức năng cơ bản, bao gồm:

\begin{itemize}
    \item \textbf{Dashboard}: Cung cấp một cái nhìn tổng quan về các hoạt động học tập gần đây, tình trạng tiến độ học tập của sinh viên và các khóa học tham gia.
    \item \textbf{Course List và Course Detail}: Cho phép sinh viên theo dõi các khóa học, xem chi tiết các bài học, và tiếp cận với các tài liệu học tập.
    \item \textbf{Đề xuất Lessons}: Dựa trên mục tiêu học tập mà sinh viên đặt ra, hệ thống sẽ tự động tìm kiếm các bài học phù hợp từ các khóa học đã đăng tải, giúp sinh viên tiếp cận đúng bài học cần thiết.
    \item \textbf{Tạo Quiz và Code Exercises}: Hệ thống có khả năng tự động tạo ra các bài quiz và bài tập lập trình dựa trên nội dung bài học, giúp sinh viên thực hành và củng cố kiến thức.
    \item \textbf{Tạo tài liệu đọc}: Hệ thống có khả năng tự động tạo ra các tài liệu đọc dựa trên nội dung bài học, giúp sinh viên có thêm tài liệu tham khảo.
    \item \textbf{Giảng viên}: Hệ thống cho phép giảng viên cập nhật learning outcomes cho các khóa học, giúp sinh viên có thể theo dõi được các mục tiêu học tập mà họ cần đạt được. Ngoài ra, giảng viên sẽ là người tải lên các tài liệu học tập để hệ thống có thể sử dụng như 1 knowledge base. Giảng viên cũng có thể theo dõi tiến độ học tập của sinh viên và đánh giá kết quả học tập của họ.
    \item \textbf{Quản trị viên}: Hệ thống cho phép quản trị viên theo dõi và quản lý các hoạt động của sinh viên và giảng viên trong hệ thống. Quản trị viên có thể xem báo cáo về các phản hồi của người dùng đối với hệ thống và thêm tài khoản cho sinh viên hoặc giảng viên. Quản trị viên cũng có thể tạo các khóa học cho giảng viên theo từng học kì. 
\end{itemize}

Các mô-đun học tập (quiz, code exercises, reading material) đã được phát triển cho phép sinh viên tham gia vào việc học theo dạng quiz, code exercises, và tài liệu đọc. Hệ thống cũng đã ghi lại quá trình học tập của sinh viên, tạo điều kiện để tiến hành báo cáo và đánh giá kết quả học tập.
\newpage
\section{Thử nghiệm hệ thống với các mô hình ngôn ngữ hiện có}

Mục đích của cuộc thử nghiệm đó là đánh giá chất lượng của lời giải thích của từng mô hình ngôn ngữ hớn dành cho người học. Mô hình tiến hành thử nghiệm:
\begin{itemize}
    \item GPT 4o
    \item GPT 4o Mini
    \item Google Gemini Pro 1.5
\end{itemize}
Dữ liệu đầu vào cho mỗi mô hình là một yêu cầu từ người dùng về việc cung cấp các tài nguyên học tập hữu ích cho việc phát triển kỹ năng. Dưới đây là kết quả trả về từ mỗi mô hình:
\begin{itemize}
    \item GPT 4o mini:
    \begin{lstlisting}[breaklines=true]
        GPT 4o mini:
        Response body
        {
          "final_response": [
            {
              "id": "b0f66328-862d-4e4d-9f5a-2bdee93490b8",
              "code": "COD001",
              "description": "A classic beginner coding challenge.",
              "explanation": "This coding challenge is a great starting point for you as it will help you practice fundamental programming skills. Engaging with such challenges will enhance your problem-solving abilities, which are essential for backend development."
            },
            {
              "id": "d3f325a9-8103-4a7b-bb26-7de45b607f9c",
              "code": "COD002",
              "description": "A simple exercise to reverse a given string.",
              "explanation": "This exercise will allow you to work on string manipulation, a critical skill in backend development. By mastering this concept, you will be better prepared to handle data processing tasks that are common in backend applications."
            },
            {
              "id": "59427089-44c4-4916-8886-52a9f6f488b1",
              "code": "CRS003",
              "description": "A comprehensive introduction to web development.",
              "explanation": "This course will provide you with a solid foundation in web development, covering essential concepts that are crucial for backend development. Understanding how the web works will empower you to create effective backend solutions."
            },
            {
              "id": "1144a6d7-a357-4e4c-b4f5-315afde72a67",
              "code": "PA001",
              "description": "A project to build a calculator using Python.",
              "explanation": "Working on this project will give you hands-on experience with Python, a popular backend language. It will help you apply your knowledge in a practical setting, enhancing your coding skills and preparing you for real-world backend tasks."
            },
            {
              "id": "7a32d80f-5313-425f-9bf2-c271084384d8",
              "code": "PA003",
              "description": "A project to create and deploy a personal portfolio website.",
              "explanation": "This project will not only allow you to practice your web development skills but also give you experience in deploying a web application. Understanding deployment is vital for backend developers, as it will help you manage and maintain applications effectively."
            }
          ]
        }
      \end{lstlisting}
    \item GPT 4o
    \begin{lstlisting}[breaklines=true]
    {
      GPT 4o:
      Response body
      "final_response": [
        {
          "id": "59427089-44c4-4916-8886-52a9f6f488b1",
          "code": "CRS003",
          "description": "A comprehensive introduction to web development.",
          "explanation": "Since you're starting your journey towards backend development, having a solid understanding of web development is essential. This course will provide you with the foundational knowledge necessary to understand how web applications are structured, which is crucial before diving into backend specifics."
        },
        {
          "id": "CRS001",
          "code": "6e906d7e-8f93-4dd8-8890-fa4b42cfcc8c",
          "description": "A beginner-friendly course to learn Python programming.",
          "explanation": "Python is a popular language for backend development due to its simplicity and effectiveness. This course is perfect for you as it will introduce you to programming concepts using Python, which is often used for backend tasks such as server scripting and data management."
        },
        {
          "id": "PA003",
          "code": "7a32d80f-5313-425f-9bf2-c271084384d8",
          "description": "A project to create and deploy a personal portfolio website.",
          "explanation": "Engaging in a project like this will give you hands-on experience with web hosting and deployment, skills that are vital for a backend developer. You'll learn how to manage servers and deploy applications, which are key components of backend development."
        },
        {
          "id": "LES002",
          "code": "1bb2500f-4275-4672-9089-fd9b382809a6",
          "description": "A lesson focused on understanding binary search.",
          "explanation": "Understanding algorithms is a critical skill for backend development, especially for optimizing data retrieval processes. This lesson will help you grasp the concept of binary search, enhancing your problem-solving skills and efficiency in handling data."
        },
        {
          "id": "PA001",
          "code": "1144a6d7-a357-4e4c-b4f5-315afde72a67",
          "description": "A project to build a calculator using Python.",
          "explanation": "This project will reinforce your Python programming skills by applying them to a practical task. It will help you understand how to implement logic and functionality, which is essential for developing backend applications."
        }
      ]
    }
    \end{lstlisting}
    \item Gemini Pro 1.5
    \begin{lstlisting}[breaklines=true]
    {
      Gemini Pro 1.5:
      Response body
      "final_response": [
        {
          "id": "b0f66328-862d-4e4d-9f5a-2bdee93490b8",
          "code": "COD001",
          "description": "A classic beginner coding challenge.",
          "explanation": "This coding challenge will help you build foundational programming skills.  It's a great way to start practicing problem-solving, a crucial skill for any programmer, especially in backend development where you'll constantly be tackling new challenges."
        },
        {
          "id": "d3f325a9-8103-4a7b-bb26-7de45b607f9c",
          "code": "COD002",
          "description": "A simple exercise to reverse a given string.",
          "explanation": "This exercise focuses on string manipulation, a core skill for backend development.  Practicing this will improve your ability to process and handle data effectively, a common task in backend applications."
        },
        {
          "id": "CRS001",
          "code": "CRS001",
          "description": "A beginner-friendly course to learn Python programming.",
          "explanation": "Since you're just starting, this course provides a foundational understanding of Python, a widely-used language for backend development.  Learning the basics will lay a solid groundwork for more advanced concepts and projects later on."
        },
        {
          "id": "LES001",
          "code": "LES001",
          "description": "An in-depth lesson on Python lists.",
          "explanation": "This lesson will build your proficiency in working with Python lists, a fundamental data structure.  Understanding list manipulation is essential for many backend programming tasks, improving your overall coding efficiency."
        },
        {
          "id": "PA001",
          "code": "PA001",
          "description": "A project to build a calculator using Python.",
          "explanation": "This project allows you to apply what you learn in a practical way. Building a calculator will solidify your understanding of fundamental programming concepts like user input, arithmetic operations, and function use within the context of a complete application."
        }
      ]
    }
    \end{lstlisting}
\end{itemize}
Đánh giá về lời giải thích các tài nguyên:
\begin{itemize}
    \item GPT-4o-mini: Các lời giải thích cho mỗi tài nguyên đều rõ ràng, nhấn mạnh lợi ích mà người học sẽ nhận được, chẳng hạn như việc nâng cao kỹ năng giải quyết vấn đề hoặc làm quen với các công cụ phát triển backend. Tuy nhiên, các lời giải thích có phần chung chung và ít tập trung vào ứng dụng thực tế của các kỹ năng học được.
    \item GPT-4o: Cung cấp những lời giải thích chi tiết hơn về lý do tại sao mỗi tài nguyên lại hữu ích cho người học, ví dụ như việc hiểu cách hoạt động của web và tầm quan trọng của các kiến thức về máy chủ và triển khai. Những lời giải thích này giúp người học hiểu rõ hơn về mục đích và ứng dụng của các tài nguyên trong việc phát triển kỹ năng backend.
    \item Gemini Pro: Cũng cung cấp những lời giải thích chi tiết, nhưng có phần thiên về việc khuyến khích người học tham gia vào các bài học thực tế. Mô hình này đưa ra những lý do cụ thể như việc học cách xử lý chuỗi (COD002) hay hiểu cách danh sách hoạt động trong Python (LES001), từ đó giúp người học nâng cao kỹ năng giải quyết vấn đề và lập trình thực tế.
\end{itemize}
Về chất lượng và chi tiết giải thích:
\begin{itemize}
    \item GPT-4o-mini: Lời giải thích trong GPT-4o-mini thường ngắn gọn và tập trung vào lý do chung chung tại sao một tài nguyên lại hữu ích, nhưng thiếu sâu sắc về cách tài nguyên đó giúp người học phát triển các kỹ năng cụ thể.
    \item GPT-4o: Các lời giải thích chi tiết hơn và cung cấp cái nhìn rõ ràng về các ứng dụng thực tế của từng tài nguyên. Chúng giúp người học nhận thức được lý do tại sao mỗi tài nguyên lại quan trọng trong hành trình học tập của họ.
    \item Gemini Pro: Cung cấp các lời giải thích rõ ràng, tuy nhiên có xu hướng tập trung vào các kỹ năng cụ thể mà người học sẽ phát triển thông qua các bài học và dự án. Gemini Pro khuyến khích người học áp dụng ngay các kỹ năng vào thực tế thông qua các dự án và bài học.
\end{itemize}
\textbf{\textit{Kết luận:}} Tất cả ba mô hình GPT-4o-mini, GPT-4o và Gemini Pro đều cung cấp các tài nguyên học tập có giá trị, nhưng mức độ chi tiết và sự rõ ràng trong lời giải thích có sự khác biệt. GPT-4o là mô hình có giải thích chi tiết và nhấn mạnh đến tầm quan trọng của việc hiểu sâu về lý thuyết, trong khi Gemini Pro có xu hướng tập trung vào thực hành nhiều hơn. GPT-4o-mini, mặc dù cung cấp các tài nguyên tương tự, nhưng các lời giải thích có phần đơn giản và ít sâu sắc hơn.

\newpage
\section{Đánh giá các hệ thống liên quan}
\begin{table}[ht]
    \resizebox{\textwidth}{!}{
    \begin{tabular}{|p{3.5cm}|p{3.5cm}|p{3.5cm}|p{3.5cm}|p{3.5cm}|p{3.5cm}|p{3.5cm}|}
    \hline
    \textbf{Tiêu chí} & \textbf{Hệ thống Adaptive LearningPath  \cite{Raj2022}} & \textbf{Hệ thống FOKE \cite{hu2024fokepersonalizedexplainableeducation}} & \textbf{CS50.ai\cite{10.1145/3626252.3630938}} & \textbf{APAS\cite{Frankford_2024}} & \textbf{LeetCode} \\ \hline
    \textbf{Cá nhân hóa} & Đề xuất lộ trình dựa trên năng lực học sinh & Dựa trên hồ sơ người học và kỹ thuật prompt engineering & Gợi ý mã nguồn, trợ lý học tập AI & Phản hồi tức thời, gợi ý cải thiện mã & Tạo bài tập nhưng ít cá nhân hóa \\ \hline
    \textbf{Tính tương thích với người dùng} & Thông qua đồ thị tri thức và tài nguyên phù hợp & Tùy chỉnh theo phong cách học tập và hành vi & Trợ lý học tập dựa trên GPT-4, trả lời câu hỏi học thuật & Phản hồi chi tiết về lỗi và cải tiến mã & Tạo bài tập dựa trên ngữ cảnh nhưng ít tùy chỉnh \\ \hline
    \textbf{Khả năng tương tác} & Chỉ có gợi ý tài liệu học tập & Tương tác qua gợi ý cấu trúc và phản hồi liên tục & Chatbot tương tác trực tiếp với sinh viên & AI-Tutor cung cấp phản hồi và gợi ý cải thiện mã & Tương tác chủ yếu thông qua bài tập và câu hỏi \\ \hline
    \textbf{Độ chính xác của đề xuất} & Tùy chỉnh theo dữ liệu thực tế & Sử dụng graph embedding và gợi ý cấu trúc & AI đưa ra phản hồi chính xác về mã nguồn & Phản hồi tức thời về mã nguồn và gợi ý cải thiện & Tạo bài tập nhưng không có giải thích chi tiết \\ \hline
    \textbf{Sự minh bạch và giải thích} & Đưa ra lý do thay đổi lộ trình học & Cung cấp giải thích rõ ràng về lộ trình học & Giải thích chi tiết về mã nguồn và phong cách lập trình & Cung cấp lý do cải thiện mã, không tiết lộ đáp án & Ít giải thích về cách tạo bài tập hoặc câu hỏi \\ \hline
    \textbf{Tính mở rộng} & Có thể mở rộng với nhiều tài nguyên học tập & Có thể mở rộng với nhiều đối tượng người học và lĩnh vực & Có thể áp dụng cho các khóa học online lớn & Có thể mở rộng cho nhiều khóa học và bài tập & Hệ thống có sẵn các bài tập cho nhiều cấp độ \\ \hline
    \textbf{Ứng dụng vào thực tế} & Có thể áp dụng cho lớp học truyền thống và trực tuyến & Ứng dụng vào nhiều môi trường học tập khác nhau & Ứng dụng trong các khóa học lớn như CS50 & Hệ thống đánh giá tự động có thể áp dụng cho lớp học và MOOCs & Ứng dụng cho các bài tập lập trình cá nhân và nhóm \\ \hline
    \textbf{Tiết kiệm thời gian cho giảng viên} & Giảm tải công việc của giảng viên & Giảm tải cho giảng viên trong việc thiết kế lộ trình học & Giảm tải cho giảng viên trong việc giải thích mã & Giảm tải cho giảng viên trong việc kiểm tra và đánh giá bài tập & Giảm tải cho giảng viên trong việc tạo bài tập \\ \hline
    \textbf{Hiệu quả học tập} & Cải thiện khả năng học của học sinh qua lộ trình cá nhân hóa & Cải thiện hiệu quả học tập nhờ cá nhân hóa sâu & Tăng cường khả năng tự học của sinh viên & Giúp sinh viên phát triển tư duy thuật toán & Hiệu quả học tập phụ thuộc vào việc tự học của người dùng \\ \hline
    \end{tabular}
    }
    \caption{Đánh giá các hệ thống liên quan}
    \end{table}
\begin{itemize}
    \item Các tiêu chí đánh giá hệ thống
    \begin{enumerate}
        \item Cá nhân hóa: Đánh giá mức độ cá nhân hóa của lộ trình học tập, phản hồi và tài nguyên học tập.
        \item Tính tương thích với người dùng: Đánh giá khả năng tương thích của hệ thống với các đặc điểm người dùng như phong cách học tập, mục tiêu học tập, và năng lực hiện tại.
        \item Khả năng tương tác: Đánh giá mức độ và tính hiệu quả của sự tương tác giữa người học và hệ thống (ví dụ: chatbot, phản hồi tức thời).
        \item Độ chính xác của đề xuất: Đánh giá khả năng của hệ thống trong việc tạo ra các đề xuất chính xác, phù hợp với người học.
        \item Sự minh bạch và giải thích: Đánh giá mức độ giải thích rõ ràng của hệ thống về các đề xuất và lý do chọn các tài liệu học tập.
        \item Tính mở rộng: Đánh giá khả năng mở rộng của hệ thống khi có nhiều người học và tài nguyên giáo dục hơn.
        \item Ứng dụng vào thực tế: Đánh giá khả năng áp dụng hệ thống vào môi trường học tập thực tế, như lớp học truyền thống, lớp học trực tuyến, hoặc hệ thống MOOCs.
        \item Tiết kiệm thời gian cho giảng viên: Đánh giá mức độ mà hệ thống giúp giảng viên giảm tải công việc như đánh giá bài tập, phản hồi học sinh, v.v.
        \item Hiệu quả học tập: Đánh giá mức độ cải thiện kết quả học tập của sinh viên khi sử dụng hệ thống.
    \end{enumerate}
\end{itemize}
\section{Các vấn đề và khó khăn trong quá trình phát triển}

\section{Kết chương}


\chapter{Tổng kết}

\section{Kết quả đạt được}

Sau một giai đoạn phát triển, hệ thống hỗ trợ học tập đã hoàn thành các tính năng cơ bản phục vụ cho việc học của sinh viên. Các phần quan trọng như trang dashboard, course list, course detail, và đề xuất bài học đã được triển khai đầy đủ, tạo ra một môi trường học tập chủ động và linh hoạt. Sinh viên có thể dễ dàng truy cập các khóa học, theo dõi tiến độ học tập, và nhận được các đề xuất học tập chính xác dựa trên mục tiêu cá nhân.

Các mô-đun học tập như quiz, code exercises, và reading material giúp sinh viên không chỉ học lý thuyết mà còn thực hành và tự kiểm tra khả năng của mình. Hệ thống cũng ghi lại các thống kê về thời gian học và tiến độ học tập, điều này giúp cung cấp các báo cáo chi tiết và đánh giá hiệu quả học tập của sinh viên.

\section{Hướng phát triển}

Mặc dù hệ thống hiện tại đã có những tính năng cơ bản, nhưng vẫn còn nhiều vấn đề cần phải giải quyết trong các giai đoạn tiếp theo. Cụ thể:

\begin{itemize}
    \item \textbf{Hoàn thiện phần authentication}: Đây là một yếu tố quan trọng để đảm bảo hệ thống có thể quản lý người dùng một cách an toàn và hiệu quả. Việc hoàn thiện chức năng này sẽ giúp phân quyền rõ ràng cho sinh viên, giảng viên, và quản trị viên (Admin).
    \item \textbf{Tối ưu hóa công nghệ và quy trình làm việc}: Nhóm cần thống nhất việc sử dụng công cụ phát triển chung như Docker và các công nghệ hỗ trợ khác, đồng thời cải thiện quy trình làm việc để đảm bảo tiến độ phát triển không bị ảnh hưởng bởi vấn đề thời gian của các thành viên trong nhóm.
    \item \textbf{Cải thiện tính chính xác của LLMs}: Hệ thống hiện tại cần được cải thiện về mặt độ chính xác trong việc sinh ra các bài học và đề xuất phù hợp. Việc tối ưu hóa thuật toán sẽ giúp hệ thống gợi ý đúng bài học hơn, đồng thời giảm thiểu việc sinh ra các bài học không liên quan.
    \item \textbf{Mở rộng các tính năng học tập}: Cần bổ sung thêm các tính năng giảng viên và Admin để tạo một môi trường học tập sinh động và hiệu quả hơn.
\end{itemize}

Nhìn chung, hệ thống đang trên đà phát triển và sẽ còn nhiều công việc cần phải hoàn thiện trong giai đoạn tiếp theo để đáp ứng được đầy đủ các nhu cầu của người dùng.

\newpage
\section*{Kế hoạch thực hiện Đồ án Tốt nghiệp}
\begin{table}[H]
    \centering
    \small 
    \begin{tabular}{|c|p{4.5cm}|p{6.5cm}|c|}
    \hline
    \textbf{Tuần} & \textbf{Nhiệm Vụ} & \textbf{Kết Quả Mong Đợi} & \textbf{Phụ Trách} \\ \hline
    1 & Authentication: Backend APIs và UI & Đăng nhập/đăng ký hoạt động & Phạm Thi \\ \hline
      & Landing Page: Thiết kế UI cơ bản & Trang giới thiệu hệ thống & Tuấn Anh \\ \hline
      & Quản lý khóa học (Giảng viên): API CRUD & API để thêm/sửa/xóa khóa học, bài học & Phương Nam \\ \hline
    2-3 & Quản lý học sinh (Giảng viên): API + UI & Giảng viên xem tiến độ và phản hồi học sinh & Phương Nam \\ \hline
        & Progress Tracking: Backend APIs & Xử lý dữ liệu học tập sinh viên & Phạm Thi \\ \hline
        & Quản lý người dùng (Admin): API CRUD & API quản lý tài khoản (giảng viên, sinh viên) & Tuấn Anh \\ \hline
    4-5-6 & Tích hợp LLM: Quiz Generation, Document Generation (Backend APIs) & API nhận input từ sinh viên, giảng viên và tạo câu hỏi trắc nghiệm tự động & Phạm Thi \\ \hline
        & Tích hợp LLM: Code Hint Module, Debugging Assistant, Explain Code Module & API cung cấp gợi ý sửa lỗi và giải thích code khi sinh viên yêu cầu & Tuấn Anh \\ \hline
        & Xử lý câu trả lời: LLM Evaluation Module & API chấm điểm câu trả lời tự động và đưa ra nhận xét & Phương Nam \\ \hline
    7-8 & Progress Tracking: UI + Biểu đồ & Giao diện hiển thị tiến độ học tập của sinh viên & Phạm Thi \\ \hline
        & Gửi Feedback (Sinh viên): Backend APIs & Gửi phản hồi về bài học/hệ thống & Phương Nam \\ \hline
        & Hiển thị Recent Activities: Backend APIs & Xử lý và lưu trữ dữ liệu hoạt động gần đây của sinh viên & Tuấn Anh \\ \hline
    9-10 & Gửi Feedback: UI tích hợp & Giao diện cho sinh viên gửi feedback & Phương Nam \\ \hline
        & Hiển thị Recent Activities: UI tích hợp & Giao diện hiển thị các hoạt động gần đây của sinh viên & Tuấn Anh \\ \hline
        & Quản lý hệ thống (Admin): Logs \& Lỗi & API và UI để admin theo dõi logs và lỗi & Tuấn Anh \\ \hline
    11-12 & Xử lý Feedback (Admin): Backend APIs & API xử lý phản hồi từ sinh viên & Phương Nam \\ \hline
        & Quản lý bài tập (Giảng viên): API CRUD & API quản lý bài tập trong khóa học & Phạm Thi \\ \hline
        & Testing Feedback Module & Kiểm tra và debug toàn bộ module Feedback & Phương Nam \\ \hline
    13 & Testing \& Debugging: Toàn bộ hệ thống & Kiểm tra, fix lỗi phát sinh trên hệ thống & Cả nhóm \\ \hline
    14 & Triển khai Backend: Deploy APIs & Backend chạy trên server test & Phạm Thi \\ \hline
        & Triển khai Frontend: Deploy giao diện & Frontend hoạt động ổn định & Tuấn Anh \\ \hline
    15 & Báo cáo \& Demo: Chuẩn bị thuyết trình & Báo cáo đầy đủ, demo hoàn thiện & Cả nhóm \\ \hline
    \end{tabular}
    \caption{Kế hoạch 15 Tuần}
\end{table}
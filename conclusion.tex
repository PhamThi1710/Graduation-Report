\chapter{Tổng kết}

\section{Kết quả đạt được}

Hệ thống học tập trực tuyến thông minh đã được thiết kế và phát triển thành công, đáp ứng các mục tiêu đề ra trong việc cung cấp một nền tảng học tập cá nhân hóa cho sinh viên ngành Khoa học máy tính và những người quan tâm đến lập trình. Các kết quả nổi bật bao gồm:

\begin{itemize}
	\item \textbf{Quản lý người dùng}: Hệ thống hỗ trợ đăng ký và đăng nhập cho ba vai trò chính: sinh viên, giảng viên và quản trị viên. Chức năng quản lý tài khoản cho phép tạo, cập nhật và quản lý thông tin người dùng một cách hiệu quả, với tích hợp đăng nhập qua email trường đại học và tài khoản Google.
	\item \textbf{Đề xuất tài nguyên học tập}: Hệ thống tự động đề xuất các khóa học, bài học và bài tập phù hợp dựa trên mục tiêu và tiến độ học tập của người dùng. Người dùng có thể linh hoạt lựa chọn phương pháp học như bài quiz, bài tập lập trình hoặc tài liệu đọc.
	\item \textbf{Tạo bài tập tự động}: Hệ thống sinh ra bài tập ở các mức độ khác nhau, từ cơ bản đến nâng cao, và tự động điều chỉnh độ khó dựa trên khả năng của người học.
	\item \textbf{Gia sư AI}: Tính năng nổi bật của hệ thống là gia sư AI, hỗ trợ giải thích mã nguồn, gợi ý tối ưu hóa mã, sửa lỗi và cung cấp phản hồi thời gian thực. 
	\item \textbf{Theo dõi tiến độ}: Hệ thống cung cấp các công cụ theo dõi và báo cáo chi tiết về tiến độ học tập, bao gồm thời gian học, số lượng bài tập hoàn thành và điểm số. Phân tích dữ liệu học tập giúp điều chỉnh lộ trình học phù hợp với từng người dùng, áp dụng thành công các phương pháp đánh giá SMART, Rubric-Based Assessment và STAR.
	\item \textbf{Tạo thêm nội dung học tập gặp khó khăn}: Hệ thống có khả năng phân tích khó khăn học tập của sinh viên và tạo tiếp nội dung bài học phù hợp, giúp giải quyết các vấn đề cụ thể và cải thiện hiệu quả học tập.
	\item \textbf{Quản lý khóa học}: Giảng viên có thể dễ dàng tạo, chỉnh sửa và quản lý khóa học, tải lên tài liệu và liên kết tài nguyên bổ sung. 
	\item \textbf{Quản trị hệ thống}: Quản trị viên có khả năng quản lý danh sách người dùng, khóa/mở tài khoản, và xử lý phản hồi từ người dùng, đảm bảo vận hành hệ thống ổn định và đáp ứng nhu cầu người dùng.
\end{itemize}

Hệ thống đã được thử nghiệm và nhận được phản hồi tích cực từ người dùng, chứng minh khả năng tối ưu hóa trải nghiệm học tập, tăng cường sự tương tác và nâng cao hiệu quả học lập trình.

\section{Hướng phát triển}

Để nâng cao chất lượng và mở rộng khả năng của hệ thống học tập trực tuyến thông minh, các hướng phát triển trong tương lai bao gồm:

\begin{itemize}
	\item \textbf{Tối ưu hóa hiệu suất hệ thống}: Vấn đề hiệu suất đang là ưu tiền hàng đầu của hệ thống, nhóm đang ưu tiên cải thiện tốc độ xử lý của LLM trong các phản hồi của gia sư AI, tạo lộ trình học tập và tạo bài tập tự động.
	\item \textbf{Tăng cường cá nhân hóa}: Phát triển các thuật toán học máy tiên tiến hơn để phân tích sâu hơn hành vi và sở thích học tập của người dùng, từ đó đề xuất các lộ trình học tập phù hợp, đồng thời nâng cấp hệ thống để các các lộ trình học tập cần được thay đổi linh hoạt hơn dựa trên năng lực của người học.
	\item \textbf{Mở rộng hỗ trợ đa ngôn ngữ lập trình}: Hiện tại, gia sư AI chủ yếu hỗ trợ một số ngôn ngữ lập trình phổ biến. Trong tương lai, hệ thống có thể mở rộng để hỗ trợ thêm các ngôn ngữ như Rust, Go, hoặc các ngôn ngữ chuyên biệt cho khoa học dữ liệu và trí tuệ nhân tạo.
	\item \textbf{Phát triển cộng đồng học tập}: Tích hợp các tính năng mạng xã hội như diễn đàn, nhóm học tập, hoặc chế độ học theo nhóm, giúp người học chia sẻ kiến thức và hỗ trợ lẫn nhau.
	\item \textbf{Mở rộng đối tượng người dùng}: Hệ thống có thể được điều chỉnh để hỗ trợ các lĩnh vực khác ngoài lập trình, như toán học, vật lý, hoặc các môn học kỹ thuật, mở rộng phạm vi ứng dụng trong giáo dục.
\end{itemize}

Những định hướng này sẽ giúp hệ thống không chỉ duy trì tính cạnh tranh mà còn trở thành một công cụ giáo dục tiên phong, đáp ứng nhu cầu học tập đa dạng và ngày càng phức tạp trong thời đại công nghệ 4.0.
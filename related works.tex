\chapter{Công trình liên quan}
\section{Hệ thống đề xuất lộ trình học}
Trong những năm qua, đã có rất nhiều nghiên cứu và phát triển liên quan đến các hệ thống đề xuất lộ trình học tập. Các hệ thống này không chỉ nhằm cá nhân hóa trải nghiệm học tập mà còn tối ưu hóa quá trình học tập dựa trên dữ liệu từ người học và tài nguyên giáo dục. Từ việc xây dựng đồ thị tri thức để tổ chức các khái niệm học tập, cho đến ứng dụng các mô hình học máy tiên tiến, các công trình này đã đặt nền móng cho việc nâng cao hiệu quả và tính thích nghi của giáo dục thông minh.
\begin{itemize}
    \item Hệ thống ở bài \cite{Raj2022} sử dụng \emph{Đồ thị tri thức} để biểu diễn mối liên hệ giữa các khái niệm (concepts) và các \emph{kết quả học tập (learning outcome)}. Dựa trên đồ thị tri thức, hệ thống sẽ dùng một giải thuật sinh đường để tạo sinh một lộ trình học tập phù hợp dựa trên đầu vào là \emph{chủ đề bắt đầu} và \emph{chủ đề mục tiêu} của học sinh. Lộ trình học cũng có thể được thay đổi dựa trên năng lực, điểm số,... (Đây là các tham số động - \emph{dynamic parameters}) của học sinh. Đặc biệt, hệ thống liên tục cập nhật độ khó của các tài nguyên học tập (\emph{learning objects – LOs}) cũng như khả năng học tập của học sinh thông qua dữ liệu được thu thập thời gian thực. Các thông số như thời gian học, điểm số đạt được, và số lần thử nghiệm (\emph{number of attempts}) đều được phân tích để điều chỉnh lộ trình sao cho phù hợp nhất với tiến độ học của từng cá nhân. Ngoài ra, hệ thống còn áp dụng các kỹ thuật gợi ý để lựa chọn các tài liệu học tập có mức độ tương thích cao với năng lực hiện tại của học sinh, đảm bảo rằng các tài liệu được gợi ý không chỉ phù hợp về mặt nội dung mà còn tối ưu hóa trải nghiệm học tập.
    \item Hệ thống FOKE \cite{hu2024fokepersonalizedexplainableeducation} hiện thực một tính năng đề xuất lộ trình học tập cá nhân hóa, tận dụng sức mạnh của các mô hình ngôn ngữ lớn, đồ thị tri thức, và kỹ thuật prompt engineering. Đầu vào của hệ thống bao gồm domain knowledge base, hồ sơ người học, và đặc tả nhiệm vụ học tập (learning task specification). Các dữ liệu này được xử lý thông qua nhiều mô-đun để tạo ra các gợi ý học tập chính xác, tương tác và có khả năng giải thích. Thành phần cốt lõi của hệ thống là \emph{Rừng tri thức (Knowledge Forest - KF)}, một phiên bản mở rộng của đồ thị tri thức. Rừng tri thức đại diện cho tri thức miền dưới dạng các cây tri thức (knowledge trees), trong đó mỗi cây thể hiện cấu trúc phân cấp của các khái niệm và mối quan hệ tương ứng. Hồ sơ người học được xây dựng theo cơ chế lập hồ sơ đa chiều (multi-dimensional user profiling), mô hình hóa các đặc điểm tĩnh (thông tin cá nhân, phong cách học tập), hành vi (hoạt động học tập, hiệu suất), và hành trình thời gian (chuỗi kỹ năng đạt được). Dựa trên hai thành phần trên, FOKE kết hợp kỹ thuật \emph{graph embedding} và các gợi ý cấu trúc (structured prompts) để tạo ra các đề xuất lộ trình học tập cá nhân hóa. Đầu ra của hệ thống bao gồm các khuyến nghị về trình tự học tập hoặc tài nguyên học tập, chẳng hạn như khóa học cần hoàn thành, các kỹ năng cần trau dồi, và các tài liệu hỗ trợ.
    \item Bài báo \cite{ng2024educationalpersonalizedlearningpath} giới thiệu một phương pháp mới để lập kế hoạch lộ trình học tập cá nhân hóa (\textit{Personalized Learning Path Planning - PLPP}) bằng cách tích hợp các mô hình ngôn ngữ lớn  và kỹ thuật thiết kế prompt engineering. Mục tiêu chính của nghiên cứu là cải thiện tính cá nhân hóa, sự thích nghi và tính minh bạch của các hệ thống giáo dục thông minh. Phương pháp đề xuất sử dụng các gợi ý được thiết kế đặc biệt để tích hợp thông tin cụ thể về người học, bao gồm kiến thức nền tảng, mục tiêu học tập và sở thích cá nhân. Các gợi ý này hướng dẫn LLMs, chẳng hạn như GPT-4 và LLama-2-70B, tạo ra các lộ trình học tập cá nhân hóa, phù hợp với nhu cầu và năng lực của từng người học. Ngoài ra, hệ thống hỗ trợ hội thoại đa lượt, cho phép mô hình đặt câu hỏi làm rõ và nhận phản hồi từ người học để tinh chỉnh các đề xuất. Ví dụ, một gợi ý có thể bao gồm: "Dựa trên hiểu biết hiện tại của học sinh về [Chủ đề], đề xuất ba khái niệm tiếp theo mà họ nên học để đạt được trình độ thành thạo trong [Môn học]. Đầu vào của hệ thống bao gồm hồ sơ người học, trạng thái kiến thức hiện tại, và mục tiêu học tập. Đầu ra là các lộ trình học tập được cá nhân hóa, có cấu trúc rõ ràng và được hỗ trợ bởi các giải thích chi tiết về thứ tự học tập được đề xuất. Ví dụ, hệ thống có thể cung cấp lời giải thích như: "Việc học khái niệm [A] trước khái niệm [B] là cần thiết để đảm bảo nền tảng vững chắc trong [Môn học]."
\end{itemize}
\section{Ứng dụng LLM vào việc giảng dạy khoa học máy tính, lập trình,...}
Trong lĩnh vực khoa học máy tính và lập trình, nơi yêu cầu người học không chỉ nắm vững lý thuyết mà còn thành thạo các kỹ năng thực hành, việc tích hợp LLMs đã mang lại nhiều lợi ích nổi bật. Các công cụ dựa trên LLMs không chỉ giúp người học giải quyết các vấn đề kỹ thuật mà còn cung cấp các phản hồi chi tiết, nâng cao khả năng tự học và giải quyết vấn đề. Một số công trình liên quan đến việc ứng dụng LLM vào việc giảng dạy lập trình có thể kể đến:
\begin{itemize}
    \item Trong bài báo \cite{10.1145/3626252.3630938}, nhóm nghiên cứu đã tích hợp AI vào khóa học CS50 tại Đại học Harvard với mục tiêu tái hiện tỷ lệ hướng dẫn lý tưởng 1:1 giữa giảng viên và sinh viên. Hệ thống, được gọi là CS50.ai, cung cấp các tính năng hỗ trợ đa dạng như:
    \begin{itemize}
        \item Giải thích mã nguồn: Sinh viên có thể làm nổi bật một đoạn mã và nhận được giải thích chi tiết bằng ngôn ngữ tự nhiên.
        \item Cải thiện phong cách mã nguồn: Công cụ style50 gợi ý các cải tiến phong cách mã lập trình dựa trên hướng dẫn phong cách của khóa học.
        \item Trợ lý học tập AI ("CS50 Duck"): Một chatbot tương tác dựa trên GPT-4, hỗ trợ trả lời các câu hỏi học thuật và hành chính liên quan đến khóa học.
    \end{itemize}
    Ngoài ra, Hệ thống được thiết kế với \emph{"pedagogical guardrails"} (đường rào sư phạm), đảm bảo rằng AI hướng dẫn thay vì cung cấp trực tiếp đáp án. Kết quả thực nghiệm cho thấy hệ thống tăng cường đáng kể khả năng tự học của sinh viên, đồng thời giảm tải cho giảng viên trong các lớp học lớn và khóa học trực tuyến quy mô lớn (MOOCs).
    \item Trong bài báo \cite{Frankford_2024}, nhóm nghiên cứu triển khai một hệ thống đánh giá lập trình tự động (Automated Programming Assessment System - APAS) Artemis, tích hợp mô hình AI-Tutor dựa trên GPT-3.5 Turbo. Hệ thống được thiết kế để hỗ trợ sinh viên lập trình bằng cách cung cấp phản hồi tức thời và hướng dẫn cải thiện mã nguồn. Tính năng nổi bật của hệ thống bao gồm:
    \begin{itemize}
        \item Phản hồi theo thời gian thực: Sinh viên có thể nhận phản hồi chi tiết về các lỗi logic, cú pháp và hiệu suất của mã nguồn.
        \item Gợi ý cải tiến mã nguồn: AI-Tutor cung cấp các lời khuyên để cải thiện chất lượng mã nguồn, như quản lý bộ nhớ tốt hơn hoặc tối ưu hóa thuật toán.
        \item Không tiết lộ đáp án: Hệ thống chỉ cung cấp hướng dẫn, tránh việc tiết lộ giải pháp trực tiếp để đảm bảo người học tự khám phá.
    \end{itemize}
    Đầu vào của hệ thống bao gồm bài tập lập trình, mã nguồn hiện tại của sinh viên, và một giải pháp mẫu do giảng viên cung cấp. Đầu ra là các phản hồi chi tiết từ AI về chất lượng mã nguồn của sinh viên và các gợi ý để cải thiện. Nghiên cứu chỉ ra rằng AI-Tutor không chỉ cải thiện chất lượng mã nguồn mà còn giúp sinh viên phát triển tư duy thuật toán. Tuy nhiên, nhóm tác giả cũng lưu ý rằng phản hồi đôi khi còn chung chung và cần cải thiện giao diện để hỗ trợ tương tác tốt hơn.
\end{itemize}
\section{Prompt engineering}
\emph{Prompt engineering} là một hướng đi khá mới do những bước phát triển lớn gần đây của mô hình ngôn ngữ nói chung và mô hình ngôn ngữ lớn nói riêng. Về khái niệm, \emph{Prompt engineering} bao gồm việc thiết kế một cách có chiến lược các hướng dẫn cụ thể cho nhiệm vụ, được gọi là \emph{prompts}, để hướng dẫn đầu ra của mô hình mà không thay đổi các tham số\cite{sahoo2024systematic}. Một số kĩ thuật tiêu biểu trong prompt engineering có thể kể đến như sau\cite{sahoo2024systematic}:
\begin{itemize}
    \item Zero-shot prompting: Mô hình được nhận môt tả của công việc mà người dùng muốn nó thực hiện trong prompt nhưng không có dữ liệu được gán nhãn (labeled) để huấn luyện. Sau đó, mô hình tận dụng kiến thức sẵn có của nó để tạo ra các dự đoán dựa trên prompt đã cung cấp để thực hiện tác vụ.
    \item Few-shot prompting: Khác với Zero-shot prompting, để thực hiện tác vụ, ngoài đặc tả về công việc sẽ được thực hiện, mô hình còn nhận được một số ví dụ cụ thể về công việc mà người dùng muốn mô hình sẽ làm.
    \item Chain-of-thought prompting\cite{DBLP:journals/corr/abs-2201-11903}: Kĩ thuật này được sử dụng nhắm cải thiện khả năng suy luận của mô hình. Thay vì chỉ đưa ra một câu trả lời trực tiếp, người dùng đưa cho mô hình cung cấp một chuỗi các bước suy nghĩ logic và có hệ thống trước khi đưa ra đáp án cuối cùng.
\end{itemize}
\section{Retrieval-Augmented Generation}
Một trong những vấn đề của LLM đó là chúng có khả năng trả về những thông tin sai lệch, và
Retrieval-Augmented Generation (hay RAG) là kĩ thuật để hạn chế vấn đề đó, khi LLM sẽ dựa trên một nguồn dữ liệu gọi là \emph{knowledge base} để đưa ra câu trả lời. Có ba giai đoạn chính trong một kĩ thuật RAG\cite{gao2024retrievalaugmentedgenerationlargelanguage} thông thường:
\begin{itemize}
    \item \emph{Indexing}: Raw data từ các nguồn khác nhau (PDF, HTML, Word, Markdown) được làm sạch và trích xuất, sau đó sẽ được chuyển đổi thành định dạng văn bản thuần nhất. Văn bản sẽ được chia ra thành các \emph{chunks} và các chunks này sẽ được biến đổi sang dạng vector và được lưu trong một vector database.
    \item \emph{Retrieval}: Khi nhận được một câu truy vấn, hệ thống sẽ biến đổi truy vấn thành vector và so sánh độ tương đồng giữa vector biểu diễn cho câu truy vấn và các đoạn văn bản đã được lập chỉ mục. Những đoạn văn bản có độ tương đồng cao nhất sẽ được chọn làm ngữ cảnh mở rộng cho mô hình.
    \item \emph{Generation}: Truy vấn và các đoạn văn bản được tổng hợp thành một prompt, sau đó mô hình ngôn ngữ lớn sẽ phản hồi cho câu prompt đó. Mô hình có thể dựa trên kiến thức sẵn có hoặc chỉ sử dụng thông tin từ các tài liệu được cung cấp.
\end{itemize}
Với sự phát triển mạnh mẽ của LLMs trong những năm gần đây, các công trình liên quan đến RAG xuất hiện nhiều. Đầu tiên, đánh chỉ mục (indexing). PDF là một trong những định dạng tài liệu phổ biến nhất trên Internet, và cũng là một trong những nguồn dữ liệu chủ yếu được sử dụng cho việc đánh chỉ mục. Một nghiên cứu được thực hiện ở bài \cite{lin2024revolutionizingretrievalaugmentedgenerationenhanced} cho thấy, công đoạn trích xuất nội dung từ một tập tin PDF (PDF parsing) có ảnh hưởng rất đáng kể đến chất lượng của việc truy vấn trong RAG. Về nguồn dữ liệu, thay vì chỉ tập trung vào văn bản là chủ yếu, có một số nghiên cứu, kĩ thuật liên quan được phát triển cho các nguồn dữ liệu khác như: Đồ thị tri thức là một dạng dữ liệu có cấu trúc, các dạng dữ liệu giả cấu trúc như PDF\cite{gao2024retrievalaugmentedgenerationlargelanguage}.
\section{Knowledge tracing/Student Profiling}
Với sự phát triển của máy tính và trí tuệ nhân tạo, nhiều công trình liên quan đến knowledge tracing/student profiling cũng đã được đề xuất và đưa ra, với mục tiêu cuối cùng là mô hình được năng lực của người học, từ đó phục vụ cho các bài toán khác trong giáo dục thông minh, đặc biệt là giáo dục thông minh. Một trong những mô hình tiêu biểu và cổ điển nhất trong lĩnh vực này đó là \emph{Bayesian knowledge tracing}, với ý tưởng đó là mô hình hóa kiến thức của học sinh bằng cách sử dụng các biến ngẫu nhiên $I_j$, trong đó:
\begin{equation}
  I_{j} = \begin{cases}
    1, & \text{Nếu học sinh thành thục kĩ năng thứ j} \\
    0, & \text{ngược lại}
  \end{cases} \label{eq:indicator}
\end{equation}
 Ngoài ra, ở trong \cite{cross_modal_2021}, nhóm tác giả đã xây dựng một mô hình knowledge tracing dựa trên một giả định đó là các khái niệm, mảng kiến thức có liên quan đến nhau về mặt ngữ nghĩa, và nếu như mức độ thành thạo của người học ở một mảng kiến thức nhất định có thể đo được, và người học có một bước tiến ở một kĩ năng, hay một khái niệm nào đó, thì ở các mảng kiến thức liên quan, người học cũng có một bước tiến nhất định tương đối.\\
Với sự phát triển của Học sâu trong những năm trở lại đây, một nghiên cứu sử dụng một mô hình với tên gọi là \emph{Deep knowledge tracing} \cite{10.5555/2969239.2969296} để giải quyết bài toán đã nêu, trong đó, ý tưởng của giải pháp đó là sử dụng mô hình \emph{Recurrent Neural Network (RNN)}. Việc sử dụng RNN cho phép ta có thể nhận diện ra được những hình mẫu phức tạp hơn trong việc mô hình hóa kiến thức của học sinh.\\
Trong những năm trở lại đây, với sự xuất hiện của Mô hình ngôn ngữ lớn, có nhiều nghiên cứu đánh giá và tìm cách ứng dụng Mô hình ngôn ngữ lớn vào trong bài toán Knowledge tracing. Trong \cite{li2024explainable}, (nhóm) tác giả đã đặt ra vấn đề rằng, do đặc tính \emph{black-box} của các mô hình học sâu, nên kết quả của chúng \emph{không thể giải thích được}, và (nhóm) tác giả đã tận dụng hai đặc tính \emph{suy luận} và  \emph{tạo sinh} của mô hình ngôn ngữ lớn để đưa ra một lời giải thích cho hoạt động đánh giá năng lực của học viên của mô hình.


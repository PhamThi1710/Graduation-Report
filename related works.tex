\chapter{Công trình liên quan}
\section{Prompt engineering}
\emph{Prompt engineering} là một hướng đi khá mới do những bước phát triển lớn gần đây của mô hình ngôn ngữ nói chung và mô hình ngôn ngữ lớn nói riêng. Về khái niệm, \emph{Prompt engineering} bao gồm việc thiết kế một cách có chiến lược các hướng dẫn cụ thể cho nhiệm vụ, được gọi là \emph{prompts}, để hướng dẫn đầu ra của mô hình mà không thay đổi các tham số\cite{sahoo2024systematic}. Một số kĩ thuật tiêu biểu trong prompt engineering có thể kể đến như sau\cite{sahoo2024systematic}:
\begin{itemize}
    \item Zero-shot prompting: Mô hình được nhận môt tả của công việc mà người dùng muốn nó thực hiện trong prompt nhưng không có dữ liệu được gán nhãn (labeled) để huấn luyện. Sau đó, mô hình tận dụng kiến thức sẵn có của nó để tạo ra các dự đoán dựa trên prompt đã cung cấp để thực hiện tác vụ.
    \item Few-shot prompting: Khác với Zero-shot prompting, để thực hiện tác vụ, ngoài đặc tả về công việc sẽ được thực hiện, mô hình còn nhận được một số ví dụ cụ thể về công việc mà người dùng muốn mô hình sẽ làm.
    \item Chain-of-thought prompting\cite{DBLP:journals/corr/abs-2201-11903}: Kĩ thuật này được sử dụng nhắm cải thiện khả năng suy luận của mô hình. Thay vì chỉ đưa ra một câu trả lời trực tiếp, người dùng đưa cho mô hình cung cấp một chuỗi các bước suy nghĩ logic và có hệ thống trước khi đưa ra đáp án cuối cùng.
\end{itemize}
\section{Retrieval-Augmentation Generation}
Một trong những vấn đề của LLM đó là chúng có khả năng trả về những thông tin sai lệch, và Retrieval-Augmentation Generation (hay RAG) là kĩ thuật để hạn chế vấn đề đó, khi LLM sẽ dựa trên một nguồn dữ liệu gọi là \emph{knowledge base} để đưa ra câu trả lời. Có ba giai đoạn chính trong một kĩ thuật RAG\cite{gao2024retrievalaugmentedgenerationlargelanguage} thông thường:
\begin{itemize}
    \item \emph{Indexing}: Raw data từ các nguồn khác nhau (PDF, HTML, Word, Markdown) được làm sạch và trích xuất, sau đó sẽ được chuyển đổi thành định dạng văn bản thuần nhất. Văn bản sẽ được chia ra thành các \emph{chunks} và các chunks này sẽ được biến đổi sang dạng vector và được lưu trong một vector database.
    \item \emph{Retrieval}: Khi nhận được một câu truy vấn, hệ thống sẽ biến đổi truy vấn thành vector và so sánh độ tương đồng giữa vector biểu diễn cho câu truy vấn và các đoạn văn bản đã được lập chỉ mục. Những đoạn văn bản có độ tương đồng cao nhất sẽ được chọn làm ngữ cảnh mở rộng cho mô hình.
    \item \emph{Generation}: Truy vấn và các đoạn văn bản được tổng hợp thành một prompt, sau đó mô hình ngôn ngữ lớn sẽ phản hồi cho câu prompt đó. Mô hình có thể dựa trên kiến thức sẵn có hoặc chỉ sử dụng thông tin từ các tài liệu được cung cấp.
\end{itemize}
Với sự phát triển mạnh mẽ của LLMs trong những năm gần đây, các công trình liên quan đến RAG xuất hiện nhiều. Đầu tiên, đánh chỉ mục (indexing). PDF là một trong những định dạng tài liệu phổ biến nhất trên Internet, và cũng là một trong những nguồn dữ liệu chủ yếu được sử dụng cho việc đánh chỉ mục. Một nghiên cứu được thực hiện ở bài \cite{lin2024revolutionizingretrievalaugmentedgenerationenhanced} cho thấy, công đoạn trích xuất nội dung từ một tập tin PDF (PDF parsing) có ảnh hưởng rất đáng kể đến chất lượng của việc truy vấn trong RAG. Về nguồn dữ liệu, thay vì chỉ tập trung vào văn bản là chủ yếu, có một số nghiên cứu, kĩ thuật liên quan được phát triển cho các nguồn dữ liệu khác như: Đồ thị tri thức là một dạng dữ liệu có cấu trúc, các dạng dữ liệu giả cấu trúc như PDF\cite{gao2024retrievalaugmentedgenerationlargelanguage}.
\section{Knowledge tracing/Student Profiling}
Với sự phát triển của máy tính và trí tuệ nhân tạo, nhiều công trình liên quan đến knowledge tracing/student profiling cũng đã được đề xuất và đưa ra, với mục tiêu cuối cùng là mô hình được năng lực của người học, từ đó phục vụ cho các bài toán khác trong giáo dục thông minh, đặc biệt là giáo dục thông minh. Một trong những mô hình tiêu biểu và cổ điển nhất trong lĩnh vực này đó là \emph{Bayesian knowledge tracing}, với ý tưởng đó là mô hình hóa kiến thức của học sinh bằng cách sử dụng các biến ngẫu nhiên $I_j$, trong đó:
\begin{equation}
  I_{j} = \begin{cases}
    1, & \text{Nếu học sinh thành thục kĩ năng thứ j} \\
    0, & \text{ngược lại}
  \end{cases} \label{eq:indicator}
\end{equation}
 Ngoài ra, ở trong \cite{cross_modal_2021}, nhóm tác giả đã xây dựng một mô hình knowledge tracing dựa trên một giả định đó là các khái niệm, mảng kiến thức có liên quan đến nhau về mặt ngữ nghĩa, và nếu như mức độ thành thạo của người học ở một mảng kiến thức nhất định có thể đo được, và người học có một bước tiến ở một kĩ năng, hay một khái niệm nào đó, thì ở các mảng kiến thức liên quan, người học cũng có một bước tiến nhất định tương đối.\\
Với sự phát triển của Học sâu trong những năm trở lại đây, một nghiên cứu sử dụng một mô hình với tên gọi là \emph{Deep knowledge tracing} \cite{10.5555/2969239.2969296} để giải quyết bài toán đã nêu, trong đó, ý tưởng của giải pháp đó là sử dụng mô hình \emph{Recurrent Neural Network (RNN)}. Việc sử dụng RNN cho phép ta có thể nhận diện ra được những hình mẫu phức tạp hơn trong việc mô hình hóa kiến thức của học sinh.\\
Trong những năm trở lại đây, với sự xuất hiện của Mô hình ngôn ngữ lớn, có nhiều nghiên cứu đánh giá và tìm cách ứng dụng Mô hình ngôn ngữ lớn vào trong bài toán Knowledge tracing. Trong \cite{li2024explainable}, (nhóm) tác giả đã đặt ra vấn đề rằng, do đặc tính \emph{black-box} của các mô hình học sâu, nên kết quả của chúng \emph{không thể giải thích được}, và (nhóm) tác giả đã tận dụng hai đặc tính \emph{suy luận} và  \emph{tạo sinh} của mô hình ngôn ngữ lớn để đưa ra một lời giải thích cho hoạt động đánh giá năng lực của học viên của mô hình.
\section{Hệ thống đề xuất lộ trình học}
Rất nhiều nghiên cứu và công trình liên quan đến hệ thống đã được đề xuất và hiện thực. Hệ thống ở bài \cite{Raj2022} sử dụng \emph{Đồ thị tri thức} để biểu diễn mối liên hệ giữa các khái niệm (concepts) và các \emph{kết quả học tập (learning outcome)}. Dựa trên đồ thị tri thức, hệ thống sẽ dùng một giải thuật sinh đường (path generation algorithm) để tạo sinh một lộ trình học tập phù hợp dựa trên đầu vào là \emph{chủ đề bắt đầu} và \emph{chủ đề mục tiêu} của học sinh. Lộ trình học cũng có thể được thay đổi dựa trên năng lực, điểm số,... (Đây là các tham số động - dynamic parameters) của học sinh. Hệ thống \emph{FOKE}\cite{hu2024fokepersonalizedexplainableeducation} hiện thực một tính năng đề xuất cho người dùng lộ trình học phù hợp dựa trên các yếu tố như năng lực, sở thích, định hướng nghề nghiệp của người học. Trong hệ thống này, một mô hình dữ liệu được cải tiến từ \emph{Đồ thị tri thức (knowledge graph)} với tên gọi là \emph{Rừng tri thức (knowledge forest)}, được sử dụng.
\section{Kết luận}

\par Chương 3 đã điểm qua một số công trình nghiên cứu quan trọng liên quan đến các công nghệ và kỹ thuật ứng dụng trong giáo dục thông minh, đặc biệt là các phương pháp sử dụng mô hình ngôn ngữ lớn (LLMs), kiến thức truy hồi (Retrieval-Augmentation Generation), theo dõi kiến thức (Knowledge Tracing), và các hệ thống đề xuất lộ trình học tập. Các phương pháp này đã đóng góp đáng kể trong việc phát triển các giải pháp giáo dục thông minh, nâng cao khả năng cá nhân hóa trong học tập và hỗ trợ giáo viên trong việc theo dõi tiến trình học tập của học sinh.

\par Tuy nhiên, mặc dù các công trình này đã đạt được những kết quả đáng kể, vẫn còn một số khoảng trống có thể khai thác trong nghiên cứu và ứng dụng thực tế:

\begin{itemize}
    \item \textbf{Tích hợp LLM với hệ thống giáo dục thông minh:} Mặc dù có sự phát triển mạnh mẽ trong việc sử dụng LLM cho các tác vụ cụ thể như sinh câu trả lời hay đánh giá năng lực, nhưng việc tích hợp chúng vào các hệ thống giáo dục toàn diện và cá nhân hóa chưa được nghiên cứu sâu. Việc kết hợp kiến thức từ LLM với các dữ liệu cụ thể của học sinh, chẳng hạn như quá trình học tập hoặc đặc điểm cá nhân, để tạo ra các lộ trình học tập tối ưu vẫn còn là một thách thức lớn.
    
    \item \textbf{Đánh giá và cải tiến khả năng suy luận của mô hình ngôn ngữ:} Các kỹ thuật như Chain-of-thought prompting đã được áp dụng để cải thiện khả năng suy luận của mô hình ngôn ngữ, nhưng chưa có nhiều nghiên cứu tập trung vào việc đánh giá và cải tiến sự chính xác trong quá trình suy luận khi áp dụng vào các bài toán giáo dục thực tiễn.

    \item \textbf{Cá nhân hóa dựa trên dữ liệu học tập thực tiễn:} Các hệ thống đề xuất lộ trình học hiện tại phần lớn dựa vào các mô hình tĩnh, như đồ thị tri thức hay các thuật toán sinh đường. Tuy nhiên, chưa có nhiều nghiên cứu tích hợp các yếu tố động như sự thay đổi trong năng lực học sinh theo thời gian, sở thích và các yếu tố ngữ cảnh khác, để xây dựng một hệ thống lộ trình học tập thực sự linh hoạt và cá nhân hóa.
\end{itemize}

Với những khoảng trống trên, công trình nghiên cứu của chúng tôi sẽ hướng đến việc kết hợp các phương pháp trên, đồng thời cải tiến tính linh hoạt và cá nhân hóa trong giáo dục thông minh. Các nghiên cứu tiếp theo sẽ tập trung vào việc phát triển các mô hình tổng hợp, kết hợp trí tuệ nhân tạo với dữ liệu học tập thực tế, để nâng cao hiệu quả học tập và khả năng dự báo tiến bộ của học sinh.

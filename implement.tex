\chapter{Hiện thực và Kiểm thử}

\section{Hiện thực hệ thống}
Hệ thống học tập được triển khai nhằm hỗ trợ sinh viên và giảng viên trong việc tổ chức và quản lý việc học tập hiệu quả hơn. Với sự hỗ trợ của Mô hình Ngôn ngữ Lớn (LLMs), hệ thống có thể tự động tạo ra các bài học đề xuất dựa trên mục tiêu học tập của sinh viên và nội dung được cung cấp bởi giảng viên.

Hệ thống được thiết kế với kiến trúc phân tách rõ ràng giữa frontend và backend, sử dụng các công nghệ hiện đại để đảm bảo tính linh hoạt, dễ mở rộng và bảo trì.

\textbf{Frontend} 

Frontend được xây dựng bằng VueJS, tích hợp Vuetify và Tailwind CSS để cung cấp giao diện người dùng trực quan, hiện đại và dễ sử dụng. Các chức năng chính bao gồm:

\begin{itemize}
    \item Quản lý dashboard của sinh viên
    \item Hiển thị danh sách khóa học, chi tiết khóa học
    \item Gửi phản hồi và nhập mục tiêu học tập
\end{itemize}

\begin{figure}[H]
    \centering
    \includegraphics[scale=0.5]{Images/Implement/frontendStructure.png}
    \caption{Cấu trúc Frontend}
\end{figure}

\textbf{Backend}

Backend được phát triển với FastAPI, kết hợp với PostgreSQL để lưu trữ dữ liệu. Kiến trúc hệ thống được tổ chức theo các thành phần chính:

\begin{itemize}
    \item Repository
    \item Provider
    \item Controller
\end{itemize}

Mỗi thành phần đảm nhiệm một vai trò cụ thể, giúp hệ thống trở nên dễ hiểu, có tính tổ chức cao và dễ bảo trì.

\begin{figure}[H]
    \centering
    \includegraphics[scale=0.5]{Images/Implement/backendStructure.png}
    \caption{Cấu trúc Backend}
\end{figure}

\section{Kiến trúc Backend}

\textbf{1. Repository} 

Repository chịu trách nhiệm quản lý các thao tác liên quan đến dữ liệu trong hệ thống. Đây là nơi thực hiện các logic liên quan đến việc truy vấn, lưu trữ và cập nhật dữ liệu từ cơ sở dữ liệu PostgreSQL.

Đặc điểm chính:

\begin{itemize}
    \item Tách biệt logic truy xuất dữ liệu khỏi các thành phần khác
    \item Hỗ trợ các thao tác CRUD (Create, Read, Update, Delete)
\end{itemize}

Ví dụ: Repository sẽ xử lý các tác vụ như:

\begin{itemize}
    \item Truy vấn thông tin sinh viên và giảng viên
    \item Cập nhật tiến độ học tập
    \item Xóa thông tin khóa học khi cần thiết
\end{itemize}

\textbf{2. Provider} 

Provider chịu trách nhiệm xử lý các logic nghiệp vụ (business logic). Đây là nơi các phép toán và xử lý phức tạp được thực hiện trước khi trả kết quả cho Controller hoặc người dùng.

Đặc điểm chính:

\begin{itemize}
    \item Không phụ thuộc vào giao diện hoặc nguồn dữ liệu cụ thể
    \item Tập trung vào logic nghiệp vụ và các phép toán quan trọng
\end{itemize}

Ví dụ: Provider sẽ:

\begin{itemize}
    \item Phân tích mục tiêu học tập của sinh viên để đề xuất bài học
    \item Tự động tạo bài học dựa trên nội dung được giảng viên cung cấp
\end{itemize}
 
\textbf{3. Controller} 

Controller là nơi tiếp nhận các yêu cầu từ người dùng thông qua API, sau đó phối hợp với Provider và Repository để thực hiện các chức năng tương ứng.

Đặc điểm chính:

\begin{itemize}
    \item Giao tiếp trực tiếp với frontend thông qua các endpoint API
    \item Chuyển đổi dữ liệu giữa frontend và các thành phần backend
\end{itemize}

Ví dụ: Controller xử lý:

\begin{itemize}
    \item Yêu cầu đăng nhập và xác thực người dùng
    \item Truy xuất danh sách khóa học và chi tiết khóa học
    \item Nhận mục tiêu học tập từ sinh viên và gửi đến Provider để xử lý
\end{itemize}

\section{Quá trình triển khai hệ thống}

\textbf{1. Tạo cấu trúc cơ sở dữ liệu} 

Cơ sở dữ liệu được thiết kế để lưu trữ các thông tin liên quan đến sinh viên, giảng viên, khóa học và bài học. Sơ đồ dưới đây minh họa cấu trúc cơ sở dữ liệu:

\begin{figure}[H]
    \centering
    \includegraphics[width=0.8\textwidth]{Images/Implement/DB_diagram.jpg}
    \caption{Cấu trúc Cơ sở dữ liệu}
\end{figure}

\textbf{2. Xây dựng API} 

Các API hỗ trợ giao tiếp giữa frontend và backend, đảm bảo rằng người dùng có thể tương tác với hệ thống một cách hiệu quả. Các chức năng chính của API bao gồm:

\begin{itemize}
    \item Quản lý khóa học
    \item Quản lý bài học
    \item Gửi và nhận phản hồi từ sinh viên
\end{itemize}

FastAPI cung cấp cơ chế xử lý yêu cầu nhanh chóng và an toàn, giúp cải thiện hiệu năng hệ thống.

\textbf{3. Quản lý bài học và đề xuất học tập} 

Hệ thống sử dụng LLMs để phân tích dữ liệu từ sinh viên và giảng viên, từ đó đưa ra các bài học phù hợp. Điều này giúp cá nhân hóa việc học tập và tăng hiệu quả của hệ thống.

Cụ thể:

\begin{itemize}
    \item Giảng viên cung cấp nội dung khóa học và mục tiêu học
    \item Sinh viên nhập mục tiêu học tập cá nhân
    \item LLMs xử lý và đề xuất các bài học tương thích
\end{itemize}

\section{Kiểm thử}

\subsection{Tổng quan về kiểm thử}

Mục đích của kiểm thử trong hệ thống: Đảm bảo rằng các tính năng hoạt động đúng, hệ thống không có lỗi và có thể chịu được tải trong quá trình sử dụng thực tế.

\subsubsection{Loại kiểm thử thực hiện}
\begin{itemize}
    \item \textbf{Kiểm thử chức năng:} Đảm bảo các chức năng của hệ thống hoạt động đúng
    \item \textbf{Kiểm thử hiệu suất:} Đảm bảo hệ thống có thể xử lý nhiều yêu cầu đồng thời
    \item \textbf{Kiểm thử bảo mật:} Đảm bảo hệ thống bảo vệ tốt các dữ liệu người dùng
    \item \textbf{Kiểm thử tích hợp:} Kiểm tra sự hoạt động đồng bộ của các module
    \item \textbf{Kiểm thử hồi quy:} Đảm bảo những thay đổi mới không làm hỏng các tính năng cũ
\end{itemize}

\subsection{Kiểm thử API}

\subsection{Mô tả chung về các kịch bản kiểm thử}

Mỗi kịch bản kiểm thử được tổ chức và theo dõi theo bảng sau đây. Các mục chính trong mỗi kịch bản kiểm thử bao gồm:
\begin{itemize}
    \item \textbf{TEST SCENARIO:} Tên kịch bản kiểm thử
    \item \textbf{Screen name/Function name:} Tên màn hình hoặc tên chức năng cần kiểm thử
    \item \textbf{Test case code:} Mã số của test case
    \item \textbf{Number of passed test cases (P):} Số lượng test case đã thành công
    \item \textbf{Number of failed test cases (F):} Số lượng test case không thành công
    \item \textbf{Number of test cases under pending (PE):} Số lượng test case đang chờ xử lý
    \item \textbf{Number of unexecuted test cases:} Số lượng test case chưa được thực hiện
    \item \textbf{Total number of test cases:} Tổng số test case cần kiểm thử
\end{itemize}

\subsubsection{Cấu trúc của một kịch bản kiểm thử}

\begin{itemize}
    \item \textbf{Test case code:} Mã của test case, giúp phân biệt các test case khác nhau
    \item \textbf{Testing Purpose:} Mục đích của việc kiểm thử
    \item \textbf{Steps:} Các bước thực hiện kiểm thử
    \item \textbf{Expected outcome:} Kết quả mong đợi sau khi thực hiện các bước kiểm thử
    \item \textbf{Browser compatibility testing:} Kiểm tra tính tương thích với các trình duyệt phổ biến
    \item \textbf{Test status report:} Báo cáo tình trạng của các lần kiểm thử
\end{itemize}

\subsubsection{Ví dụ về một kịch bản kiểm thử trong bảng Excel}
\begin{figure}[H]
    \centering
    \includegraphics[scale=0.5]{Images/Implement/testScenario.png}
    \caption{Ví dụ về một kịch bản kiểm thử}
\end{figure}
\subsubsection{Mục tiêu của các kịch bản kiểm thử}
\begin{itemize}
    \item \textbf{Kiểm thử chức năng}: Đảm bảo các API hoạt động đúng với các yêu cầu và dữ liệu đầu vào hợp lệ hoặc không hợp lệ. 
    \item \textbf{Kiểm thử tương thích trình duyệt}: Đảm bảo hệ thống hoạt động ổn định trên các trình duyệt phổ biến.
    \item \textbf{Kiểm thử độ tin cậy}: Kiểm tra xem các API có thể xử lý được các tình huống thực tế và đưa ra kết quả chính xác.
\end{itemize}
\subsubsection{Kết quả kiểm thử API}

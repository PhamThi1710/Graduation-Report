\chapter{Kiến thức nền tảng}
\section{Kiến thức nền tảng về giáo dục và giáo dục thông minh}
\subsection{Khái quát về giáo dục}
\par Trong phần này, ta sẽ giới thiệu khái niệm cơ bản về giáo dục. Giáo dục là quá trình truyền đạt và tiếp thu kiến thức, kỹ năng và giá trị, giúp con người phát triển toàn diện về trí tuệ, thể chất và đạo đức. Mục tiêu chính của giáo dục là trang bị cho con người những kiến thức cơ bản cần thiết để hòa nhập vào xã hội, đồng thời phát triển các kỹ năng tư duy, giải quyết vấn đề và sáng tạo.

\par Bên cạnh đó, giáo dục còn hướng đến việc bồi dưỡng phẩm chất đạo đức, khuyến khích mỗi cá nhân phát triển tiềm năng riêng biệt của mình. Đây chính là nền tảng để mỗi người có thể xây dựng sự nghiệp và cuộc sống thành công.

\par Trong bối cảnh thế giới không ngừng thay đổi, giáo dục ngày nay không chỉ cung cấp kiến thức mà còn giúp con người thích nghi với những thách thức mới, phát triển khả năng học tập suốt đời và duy trì sự phát triển liên tục qua mọi giai đoạn cuộc sống.

\par Về lý thuyết giáo dục, có nhiều học thuyết liên quan đến lĩnh vực này, kết hợp từ nhiều lĩnh vực khác nhau như tâm lý học và thần kinh học:
\begin{itemize}
\item \emph{Học tập kiến tạo (Constructivism): }Phương pháp học tập này cho rằng con người tự xây dựng sự hiểu biết và kiến thức của mình thông qua trải nghiệm và việc phản ánh lại các trải nghiệm đó. Thay vì tiếp thu thông tin một cách thụ động, người học chủ động tạo ra ý nghĩa thông qua việc liên kết thông tin mới với kiến thức đã có trước đó.
\item \emph{Thuyết tải nhận thức (Cognitive load theory): }Được phát triển bởi John Sweller vào những năm 1980, lý thuyết này tập trung vào cách bộ não xử lý và lưu trữ thông tin trong quá trình học tập, đặc biệt là những yếu tố ảnh hưởng đến hiệu quả học tập. Thuyết này nhấn mạnh rằng bộ não con người có giới hạn về lượng thông tin có thể xử lý cùng lúc trong trí nhớ ngắn hạn.
\end{itemize}
\subsection{Khái quát về giáo dục thông minh}
\par Tuy không có một định nghĩa chính thức và rõ ràng cho thuật ngữ \textbf{“giáo dục thông minh”}, trong báo cáo này, thuật ngữ \textbf{"thông minh"} được hiểu là cách thức thực hiện mọi hoạt động một cách hiệu quả và hữu ích. Mục tiêu của giáo dục thông minh là bồi dưỡng con người trên nhiều mặt, từ kiến thức đến đạo đức, với phương thức giúp học sinh tiếp thu kiến thức một cách hiệu quả nhất có thể.

\par Giáo dục thông minh không chỉ liên quan đến việc dạy và học mà còn phải thích ứng với nhu cầu, khả năng, sở thích và điểm mạnh, yếu của từng học sinh. Việc cá nhân hóa quá trình học tập đóng vai trò quan trọng, vì mỗi học sinh có cách tiếp thu khác nhau. Công nghệ trí tuệ nhân tạo (AI) hiện nay đã tạo ra nhiều cơ hội để phát triển giáo dục thông minh, đặc biệt là qua phương pháp học tập cá nhân hóa (personalized learning), nơi mà mỗi học sinh có thể học theo một tốc độ, phong cách và nội dung phù hợp với bản thân.

\par Học tập cá nhân hóa nhằm tối đa hóa hiệu quả học tập bằng cách điều chỉnh chương trình học và phương pháp giảng dạy để phù hợp với nhu cầu và mục tiêu riêng biệt của mỗi học sinh, thay vì áp dụng một chương trình chung cho tất cả.
\section{Mô hình ngôn ngữ}
\par \emph{Mô hình ngôn ngữ (language model)} có thể được hiểu ở nhiều góc độ khác nhau, nhưng nhìn chung, một mô hình ngôn ngữ là một hệ thống được thiết kế để hiểu và tạo ra ngôn ngữ tự nhiên của con người. Trong lĩnh vực machine learning, mô hình ngôn ngữ được định nghĩa rõ ràng hơn là một mô hình xác suất của một ngôn ngữ tự nhiên, có khả năng dự đoán một từ hoặc một chuỗi từ dựa trên những từ đã xuất hiện trước đó.

\par Trong lịch sử phát triển của lĩnh vực xử lý ngôn ngữ tự nhiên (NLP), đã có nhiều loại mô hình ngôn ngữ khác nhau. Dưới đây là các loại mô hình ngôn ngữ chính:
\begin{itemize}
    \item \textbf{Mô hình ngôn ngữ xác suất:} Là loại mô hình ngôn ngữ là trong đó, mỗi một chuỗi từ ngữ nhất định sẽ được gán một xác suất cụ thể. Bài toán chính mà dạng mô hình này giải quyết đó là tính \emph{khả năng xảy ra (likelihood)} của một chuỗi các từ vựng bất kì, câu hỏi đặt ra thường là "Liệu một chuỗi các từ $w_1, w_2,..., w_n$ có khả năng xuất hiện là bao nhiêu?". Một ví dụ điển hình cho dạng mô hình ngôn ngữ xác suất đó là mô hình ngôn ngữ n-gram. Theo mô hình ngôn ngữ này, thì xác suất của một từ xuất hiện phụ thuộc vào các từ trước đó:
    \begin{align*}
       P(w_1, w_2, \ldots, w_n) = \prod_{i=1}^{n} P(w_i \mid w_{i-(n-1)}, \ldots, w_{i-1})     
    \end{align*}
    \item \textbf{Mô hình ngôn ngữ dựa trên Neural network: } là một loại mô hình ngôn ngữ sử dụng neural network để hiểu và sinh ra ngôn ngữ tự nhiên của con người. Sức mạnh của neural network nằm ở chỗ là nó có thể nhận diện được các pattern phức tạp, do đó, mô hình dạng này giúp thực hiện nhiều nhiệm vụ xử lý ngôn ngữ tự nhiên (NLP) một cách hiệu quả hơn so với các mô hình thống kê truyền thống đã nói ở trên.
    \item \textbf{Mô hình ngôn ngữ lớn:} Dựa trên khái niệm về mô hình ngôn ngữ, ta có khái niệm cho \emph{Mô hình ngôn ngữ lớn (Large language model)}. Mô hình ngôn ngữ lớn là gì? Nó khác gì so với một mô hình ngôn ngữ? Làm sao để biết được một mô hình ngôn ngữ là 'lớn'? Một mô hình ngôn ngữ lớn là một mô hình ngôn ngữ dựa trên \emph{Deep neural network} và được huấn luyện bằng một tập dữ liệu ngôn ngữ khổng lồ. Việc đánh giá một mô hình là to hay nhỏ có nhiều khía cạnh, có thể là quy mô của mô hình hoặc là khả năng. Về quy mô, mô hình ngôn ngữ lớn, đúng như tên gọi, có kích thước rất lớn, với số lượng tham số lên đến con số hàng tỉ, hoặc hàng trăm tỉ. Một ví dụ tiêu biểu có thể kể đến như mô hình GPT-3 của OpenAI có khoảng 175 tỉ tham số \cite{brown2020languagemodelsfewshotlearners}. Một mô hình ngôn ngữ lớn có thể thực hiện được nhiều tác vụ hơn, liên quan đến suy luận, giải quyết vấn đề, những thứ mà các mô hình ngôn ngữ truyền thống gần như không thể thực hiện được.
\end{itemize}
\section{Các thuật ngữ kỹ thuật liên quan đến giáo dục thông minh}
\subsection{Đồ thị tri thức (Knowledge graph)}
Trước khi nói về thuật ngữ \emph{Đồ thị tri thức}, ta hãy nhìn sơ qua về thuật ngữ \emph{Đồ thị}. Ở đây, nghĩa của từ đồ thị được hiểu theo định nghĩa của nó ở trong \emph{Lý thuyết đồ thị (graph theory)} là một nhánh nghiên cứu của \emph{Toán rời rạc (Discrete mathematics)}: Một đồ thị (graph) là một cấu trúc toán học được sử dụng để mô tả mối quan hệ giữa các đối tượng (được gọi là đỉnh hoặc nút). Các đối tượng này được kết nối với nhau thông qua các cạnh (hay còn gọi là cung) biểu diễn mối quan hệ giữa chúng. Từ định nghĩa của đồ thị, Trong bài báo cáo này, nhóm xin đưa ra định nghĩa của \emph{Đồ thị tri thức (knowledge graph)}: là một dạng thể hiện có cấu trúc, cụ thể hơn là thể hiện bằng đồ thị, trong đó các đỉnh là các thực thể trong thế giới thực như: một đối tượng, một khái niệm, một sự kiện, và các cạnh là mối quan hệ giữa chúng\cite{ibm_knowledge_graph}.
\subsection{Knowledge tracing}
\emph{Knowledge tracing} là bài toán mô hình hóa kiến thức của học sinh theo thời gian để ta có thể dự đoán chính xác cách học sinh sẽ thực hiện trong các tương tác sau này \cite{10.5555/2969239.2969296}. Nhờ có \emph{Knowledge tracing}, giảng viên/người giảng dạy/hệ thống có thể kiểm tra được tiến độ học tập cũng như năng lực của mỗi học sinh \emph{một cách hiệu quả hơn} bằng cách theo dỡi tương tác của học viên với các học liệu trực tuyến (tài liệu, các câu hỏi, bài kiểm tra,...)\cite{10.1145/3569576}.\\

\chapter{Giới thiệu về đề tài}
\section{Động lực}
Trong bối cảnh toàn cầu hóa và sự phát triển không ngừng của công nghệ, việc nâng cao chất lượng giáo dục, đặc biệt trong lĩnh vực lập trình, ngày càng trở nên quan trọng. Một trong những thách thức lớn hiện nay là làm thế nào để cá nhân hóa quá trình học lập trình, đáp ứng sự đa dạng trong nhu cầu và khả năng của từng học viên. Các hệ thống giáo dục truyền thống thường gặp khó khăn trong việc tạo ra lộ trình học phù hợp với từng học viên, đặc biệt là trong lập trình, nơi mà khả năng tư duy và phong cách học của mỗi học viên đều khác nhau.

Sự phát triển của công nghệ trí tuệ nhân tạo (AI) và mô hình ngôn ngữ lớn (LLM) đang mở ra cơ hội lớn để xây dựng các hệ thống dạy học lập trình hiệu quả hơn. LLM, với khả năng hiểu và sinh ngôn ngữ tự nhiên, không chỉ giúp học viên giải quyết các vấn đề lập trình nhanh chóng mà còn hỗ trợ cá nhân hóa học tập. LLM có thể cung cấp tài liệu học tập và bài giảng được tùy chỉnh, cũng như hỗ trợ giải quyết bài tập lập trình, từ đó cải thiện hiệu quả học tập.

Tuy nhiên, sự phát triển mạnh mẽ của AI cũng gây ra mối lo ngại về việc sinh viên sử dụng AI để làm bài tập mà không thực sự học hỏi. Điều này đặt ra câu hỏi liệu AI có thể giúp sinh viên tự tìm ra lời giải cho các vấn đề thay vì đưa ra câu trả lời trực tiếp, giúp họ phát triển kỹ năng tư duy giải quyết vấn đề một cách hiệu quả.
\par Việc chọn giáo dục lập trình làm lĩnh vực nghiên cứu và ứng dụng hệ thống có hai lý do chính:
\begin{itemize}
    \item Thứ nhất, dạy học lập trình ngày càng trở nên phổ biến trong thời đại internet, giúp học viên phát triển tư duy logic và khả năng giải quyết vấn đề, vì nó yêu cầu phân tích và tư duy hệ thống.
    \item Thứ hai, lĩnh vực giáo dục lập trình rất gần gũi với ngành khoa học và kỹ thuật máy tính, nơi mà nhóm nghiên cứu đang theo học, tạo ra sự liên kết thực tiễn giữa nghiên cứu và ứng dụng trong thực tế.
\end{itemize}
\section{Mục tiêu}
Từ động lực đã nêu, mục tiêu của đề tài này gồm các điểm chính như sau:
\begin{itemize}
    \item \textbf{Xây dựng hệ thống học tập cá nhân hóa: }Hệ thống sẽ tập trung vào việc cá nhân hóa trải nghiệm học cho từng học viên, phù hợp với khả năng, trình độ, và nhu cầu học tập riêng biệt. Mỗi học viên có một lộ trình học và cách giải thích riêng, giúp họ tiến bộ một cách hiệu quả nhất. Với khả năng phân tích và hiểu ngữ cảnh của LLM, hệ thống có thể đưa ra các phản hồi phù hợp với nhu cầu học tập của từng học viên.
    \item \textbf{Xác định tính khả thi của việc tích hợp LLM trong giáo dục: }Đánh giá khả năng tích hợp mô hình ngôn ngữ lớn vào hệ thống giáo dục, đặc biệt là trong bối cảnh giáo dục lập trình. Chúng tôi sẽ nghiên cứu tính hiệu quả của LLM trong việc cá nhân hóa và nâng cao khả năng tiếp thu kiến thức của học viên.
    \item \textbf{Giải quyết vấn đề lạm dụng LLM trong giải bài tập: }Hiện nay, sinh viên có thể lợi dụng LLM để giải bài tập lập trình mà không thực sự học. Mục tiêu của đề tài là nghiên cứu xem liệu có thể khiến LLM không trực tiếp đưa ra lời giải cho học sinh mà thay vào đó là cung cấp các câu hỏi gợi mở hoặc hướng dẫn giúp học viên tự giải quyết vấn đề. Điều này nhằm phát triển tư duy giải quyết vấn đề của học viên, thay vì chỉ đưa ra câu trả lời.
\end{itemize}
\section{Phạm vi}
\par Đề tài này sẽ được thực hiện trong khoảng thời gian 7-8 tháng, bao gồm hai học kỳ tại trường đại học. Phạm vi nghiên cứu chủ yếu tập trung vào lĩnh vực giáo dục công nghệ thông tin, đặc biệt là trong việc giảng dạy lập trình. Cụ thể, hệ thống sẽ được xây dựng để hỗ trợ việc học lập trình cho sinh viên, thông qua việc cá nhân hóa các bài giảng, hướng dẫn và hỗ trợ học tập dựa trên các mô hình ngôn ngữ lớn.

\par Về mặt công nghệ, nhóm thực hiện đề tài sẽ không xây dựng lại các mô hình ngôn ngữ lớn từ đầu mà sẽ tận dụng các mô hình hiện có, chẳng hạn như GPT, Gemini, hoặc các mô hình tương tự, để khai thác và tùy chỉnh chúng nhằm đáp ứng các yêu cầu cụ thể của hệ thống. Mô hình ngôn ngữ này sẽ được sử dụng để cung cấp các phản hồi tự động, đề xuất bài học, tạo các bài kiểm tra và giúp cá nhân hóa trải nghiệm học lập trình của sinh viên.

\par Mặc dù hệ thống có thể áp dụng cho các lĩnh vực giáo dục khác, trong phạm vi của đề tài này, mục tiêu chính là tập trung vào việc ứng dụng trong lĩnh vực lập trình và công nghệ thông tin, nhằm hỗ trợ việc giảng dạy và học tập môn học này.
\section{Ý nghĩa của đề tài}
\subsection{Ý nghĩa của dự án dưới góc nhìn thực tiễn}
\begin{itemize}
    \item Dự án này mang lại ý nghĩa thực tiễn quan trọng trong việc áp dụng công nghệ AI vào giáo dục, cụ thể là trong việc cá nhân hóa quá trình học tập của học viên. Nhờ vào AI và LLM, hệ thống học tập có thể tạo ra một trải nghiệm học tập hấp dẫn và hiệu quả hơn cho học viên, từ đó giúp tăng động lực học tập.
    \item AI có thể đóng vai trò như một trợ giảng 1-1, giúp giảm tải áp lực cho giáo viên. AI sẽ hỗ trợ trả lời các câu hỏi cơ bản, đánh giá và chấm điểm tự động cho các bài tập, giúp giáo viên có thể tập trung vào việc hỗ trợ học viên ở mức độ sâu hơn. Điều này không chỉ cải thiện hiệu quả giảng dạy mà còn tạo ra một môi trường học tập linh hoạt và phù hợp với từng cá nhân học viên.
\end{itemize}
\subsection{Ý nghĩa của dự án theo góc nhìn khoa học}
\begin{itemize}
    \item Dự án này có ý nghĩa khoa học đáng kể trong việc ứng dụng AI và LLM vào giáo dục, đặc biệt là trong giáo dục thông minh. Việc tích hợp LLM sẽ không chỉ mở rộng khả năng của các hệ thống dạy học mà còn đánh giá được hiệu quả của chúng trong việc nâng cao khả năng giải quyết vấn đề và học lập trình.
    \item Dự án sẽ chứng minh tính ứng dụng của LLM trong việc thực hiện các tác vụ lập trình và giải quyết vấn đề, đồng thời khai thác các lý thuyết giáo dục trong khoa học nhận thức, như lý thuyết tải nhận thức (cognitive load theory) và học thuyết kiến tạo (constructivism). Việc áp dụng LLM vào giáo dục có thể giúp học viên không chỉ tiếp thu kiến thức mà còn phát triển khả năng tự học và tư duy phản biện.
\end{itemize}

\section{Cấu trúc báo cáo}

Báo cáo này được tổ chức thành 10 chương chính, mỗi chương trình bày một phần quan trọng trong quá trình thực hiện và phát triển hệ thống. Cấu trúc báo cáo như sau:

\begin{itemize}
    \item \textbf{Chương 1 – Giới thiệu về đề tài}: Cung cấp cái nhìn tổng quan về động lực, mục tiêu, phạm vi và ý nghĩa của đề tài cả trong thực tiễn và khoa học. Phần này cũng giới thiệu cấu trúc báo cáo để làm rõ cách thức bố trí và tiếp cận các nội dung.
    
    \item \textbf{Chương 2 – Kiến thức nền tảng}: Trình bày các kiến thức cơ bản liên quan đến giáo dục, giáo dục thông minh, mô hình ngôn ngữ, và các công nghệ được sử dụng như LangChain, LlamaIndex, React, FastAPI, cùng các khái niệm kỹ thuật như đồ thị tri thức và knowledge tracing.
    
    \item \textbf{Chương 3 – Công trình liên quan}: Phân tích các nghiên cứu và hệ thống liên quan, bao gồm hệ thống đề xuất lộ trình học, ứng dụng mô hình ngôn ngữ lớn (LLM), prompt engineering, retrieval-augmentation generation và knowledge tracing, nhằm xác định khoảng trống nghiên cứu và cơ hội cải tiến.
    
    \item \textbf{Chương 4 – Hệ thống đề xuất}: Mô tả chi tiết hệ thống đề xuất, bao gồm khảo sát yêu cầu, người dùng mục tiêu, các chức năng như quản lý người dùng, đề xuất tài nguyên học tập, tạo bài tập, trợ lý AI, theo dõi tiến độ và quản lý khóa học, cùng với yêu cầu phi chức năng và dữ liệu.
    
    \item \textbf{Chương 5 – Phân tích và thiết kế}: Trình bày quá trình phân tích và thiết kế hệ thống, bao gồm mô hình hóa quy trình nghiệp vụ, biểu đồ use case, kiến trúc hệ thống, thiết kế cơ sở dữ liệu (EER diagram, lược đồ quan hệ), biểu đồ lớp, sitemap và quy trình đề xuất lộ trình học tập cá nhân hóa.
    
    \item \textbf{Chương 6 – Hiện thực}: Mô tả quá trình triển khai hệ thống, kiến trúc backend, các chức năng đã thực hiện như quản lý tài khoản, khóa học, bài học, bài tập, lộ trình học tập, theo dõi tiến độ, và các phương pháp như SMART, Rubric-Based Assessment, STAR trong thiết kế mục tiêu và đánh giá.
    
    \item \textbf{Chương 7 – Giao diện người dùng}: Giới thiệu các giao diện chính của hệ thống, bao gồm landing page, đăng nhập, hồ sơ cá nhân, giao diện cho sinh viên, giảng viên và admin, với các chức năng như quản lý khóa học, bài học, bài kiểm tra và phản hồi.
    
    \item \textbf{Chương 8 – Kiểm thử hệ thống}: Mô tả quá trình kiểm thử, các kịch bản kiểm thử, API được kiểm tra, kết quả kiểm thử và các hạn chế gặp phải trong quá trình kiểm thử hệ thống.
    
    \item \textbf{Chương 9 – Đánh giá hệ thống}: Đánh giá tính năng, hiệu quả của hệ thống, so sánh với các mô hình ngôn ngữ hiện có, phân tích các hệ thống liên quan, và thảo luận các vấn đề, khó khăn trong quá trình phát triển.
    
    \item \textbf{Chương 10 – Tổng kết}: Tóm tắt kết quả đạt được, thảo luận các hạn chế của hệ thống và đề xuất các hướng phát triển trong tương lai.
\end{itemize}

